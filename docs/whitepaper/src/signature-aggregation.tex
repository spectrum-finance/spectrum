In this section we describe our approach to the following problem.
The naive approach to writing the consensus values on\
the blockchain in a verifiable way would be simply write the resulting values together with the signatures from every\
node which successfully participated in the consensus protocol.
Spectrum consensus groups can contain thousands of nodes.
If one takes Schnorr signature scheme~\cite{Schnorr1991} with 256-bit keys, every signature is 64 bytes long.
That means thousands of kilobytes of data needed to be written on the blockchain and consuming\
valuable storage space, not speaking on the computational efforts from the blockchain\
validating node to actually verify all these signatures.
Therefore, in these circumstances, the signature aggregation method is mandatory.

The aggregation allows one to write a single shorter signature instead of the list of signatures\
while preserving similar security level.
There are few signature aggregation schemes for the sigma-protocol based signatures, such as CoSi~\cite{Syta2016}\
and MuSig~\cite{Itakura1983APC}.
These protocols perform extremely well if all the keys of the predefined set of co-signers are included\
in the resulting signature generation.
In this case instead of having thousands of separate signatures one has only one of\
the size of single Schnorr signature.
But this is not the case with many realistic situations with large consensus groups (such as Spectrum).
It would be too optimistic to assume that all the nodes are always online,\
and every single node is following the protocol honestly to every letter.
One needs the mechanism to process these failures.
Whereas CoSi proposes the method to process such failures, it comes at cost of significant increase\
in the size of the resulting signature.
Our scheme relies on the similar ideas, however we tend to provide better scaling with faulty nodes and\
more compact constructions than the original CoSi.

In short, we construct a compact aggregated signature scheme with potential node failures based\
on standard cryptographic primitives.
It must have constant small size in the absence of failures and provide reasonably small space and\
computational overheads in the presence of failures.
The signing protocol must be performed in a distributed fashion\
providing defence from the malicious co-signers.

\subsubsection{General Overview}

We start with the MuSig scheme and modify it to the meet the criteria listed above.
We assume the Discrete Logarithm group to be the subgroup of the elliptic curve as usual.
That is, elliptic curve is defined over finite field, we consider subgroup of its points\
with coordinates in this field of prime order with\
fixed generator $g$ and identity element being the point at infinity if the curve\
is written in the form ${y^2=f(x)}$, $f$ is the third degree polynomial.
Nothing prevents one from using another group with hard discrete logarithm problem.
We use multiplicative notation for the group operation,\
and the group elements except for generator are written in capital letters.
The secret keys are the integers modulo group order, we will denote them by lowercase letters.
$H$ is the cryptographic hash function.
When we write ${H(A,B)}$, we assume that there is a deterministic way of serializing the tuple ${(A,B)}$,\
and this serialization is used as an argument for $H$.
The public key corresponding to the private key $x$ is the group element ${X=g^x}$.

Any interactive sigma-protocol consists of three stages in strict order:\
commitment (when one or more group elements are sent from prover to verifier),\
challenge (when the random number is sent from verifier to prover), response (when one or more\
numbers calculated from the previous stages and the secret key are sent from prover to verifier).
This triple constitutes the Proof-of-Knowledge of the secret key.
To turn the interactive protocol into a non-interactive one, Fiat-Shamir heuristic is used, where the challenge\
is replaced by the hash value of all the preceding public data.

The takeaways from this setting, which are important for the understanding of our construction are the following:
\begin{itemize}
    \item In case of $n$ nodes one must have $n$ commitments to aggregate and the list should not be changed\
    till the end of the protocol.
    \item As the commitments from different nodes come at potentially different time,\
    there can be an attack on this stage.
    Say, one node does not pick the commitment based on random, but rather calculates it based on the\
    commitments received from the other nodes.
    This kind of attack is known as $k$-list attack, as to forge the\
    upcoming signatures the malicious node solves the $k$-list problem, which is quite possible with a sufficient\
    amount of data.
    To exclude this possibility one needs all the nodes to \enquote{commit to Schnorr commitment} beforehand.
    One can use hash function with no homomorphic properties for that purpose.
    \item All the steps are strictly sequential.
    Hence, every stage must complete with the full aggregation of
    individual contributions.
    There does not seem to be a simple way to perform it fully asynchronously.
    \item Instead of the last step (response) it is sufficient to provide the proof of knowledge for the response.
    This brings no additional value to the conventional signatures, but it helps with the processing\
    of the node failures during the execution.
    Namely, the consensus group may demonstrate that somebody in the group knew the discrete\
    logarithms of the commitments not accounted for in the response stage.
    Therefore, the group as a whole could
    compute the full response if the failure had not occurred.
    \item There must be a way to count the failures above, such that the signature verifier could decide whether it\
    tolerates this number or not.
\end{itemize}

\subsubsection{Aggregation Rounds and Structures}
Here we list the overall structure of aggregation to give a grasp on the overall process.
The detailed explanation is presented below:
\begin{itemize}
    \item [] \textbf{Round 1.} Collect Commitments for Schnorr commitments.
    Structure: list of hashes of elliptic curve points.
    Distribute all the hashes after the aggregation.
    \item [] \textbf{Round 2.} Collect and aggregate Schnorr commitments.
    Structures: list of signatures (proofs of discrete\
    logarithms for the commitments) together with Schnorr commitments.
    Distribute among all the nodes.
    Upon receiving every node verifies that the hashes of the points are those\
    provided on round 1, and verifies the proofs of discrete logarithms.
    The commitments with the checks passed are aggregated to get the overall commitment.
    It is used to compute the challenge and the individual responses in
    the sigma--protocol.
    \item [] \textbf{Round 3.} Collect and aggregate the responses.
    Structure: list of individual responses.
    Upon receiving every individual response is verified.
    The responses which passed the verification are added together.
    If the response is invalid or missing, the corresponding discrete logarithm proof\
    from round 2 is appended to the output.

    \item [] \textbf{Output.} Aggregate signature $(Y,z)$ together with the set\
    \[
        \{(Y_i, DlogProof(Y_i)\}\,,
    \]
    where $i$ runs over the set of nodes which have not provided valid responses.
\end{itemize}

\subsubsection{Signature Generation}
The signature generation algorithm is as follows:
\begin{enumerate}
    \item Each signer computes ${a_i\leftarrow(X_1,X_2,\dots,X_n;X_i)}$ and the aggregate\
    public key ${\tilde{X}\leftarrow\prod_i X_i^{a_i}}$.
    \item Each signer generates a pair $Y_i=g^{y_i}$ to commit to, commitment ${t_i\leftarrow H(Y_i)}$\
    and the signature $\sigma_i$ of some predefined message with secret key $y_i$.
    \item The commitments $t_i$ are aggregated in the list $L_1$.
    \item After every participating co--signer received $L_1$, the tuples ${(Y_i,\sigma_i)}$ are aggregated in the list\
    $L_2$.
    \item Upon receiving the tuple ${(Y_i,\sigma_i)}$, verify ${t_i = H(Y_i)}$, and verify that $\sigma_i$ is a valid\
    signature corresponding to $Y_i$.
    The failed records are excluded from $L_2$, the next steps and communication round.
    \item Every node computes the aggregate commitment ${Y=\prod_i Y_i}$ using all the valid records in $L_2$.
    \item Every node computes the challenge ${c\leftarrow H(\tilde{X}, Y, m)}$\
    and the responses ${z_i\leftarrow y_i + ca_ix_i}$.
    \item The responses $z_i$ are aggregated into list $L_3$.
    \item Initialize $z\leftarrow 0$ and empty set $R\leftarrow\{\}$.
    \item Upon receiving the response $z_i$, verify that ${g^{z_i} = Y_i X_i^{a_ic}}$.
    \item If this is the case, set ${z\leftarrow z+z_i}$.
    Otherwise, insert corresponding entry from $L_2$ in $R$ as ${(i, Y_i, \sigma_i)}$.
    \item Output the triple ${(Y,z,R)}$.
\end{enumerate}

\subsubsection{Signature Verification}
The signature verification is carried out as follows:
\begin{enumerate}
    \item Compute ${a_i\leftarrow H(X_1,X_2,\dots,X_n;X_i)}$.
    \item Compute ${\tilde{X}\leftarrow\prod_i X_i^{a_i}}$.
    \item Compute ${X' = \prod_{i\notin R.0}X_i^{a_i}}$.
    \item Compute ${Y' = \prod_{i\in R.0} Y_i}$.
    \item Compute ${c\leftarrow H(\tilde{X}, Y, m)}$.
    \item Verify ${g^z=X'^cYY'^{-1}}$.
    \item Verify all of ${\sigma_i\in R.2}$ with respect to ${Y_i\in R.1}$.
    \item Compare ${(n-k)}$ (where $k$ is the size of $R$) with the required threshold value.
\end{enumerate}

\subsubsection{Instantiation of Signature Aggregation}\label{subsubsec:instantiation-of-signature-aggregation}

We instantiate our signature aggregation protocol on top of Handel~\cite{bégassat2019handel},\
a Byzantine-tolerant aggregation protocol that allows for the quick aggregation of cryptographic\
signatures over a WAN\@.
Handel has polylogarithmic time, communication and processing complexity.

Our signature aggregation protocol involves aggregation of three lists: $L1$, $L2$ and $L3$.
As long as Handel requires that the partial aggregation function satisfies both \empth{commutativity}\
and \empth{associativity} conditions we have to replace \empth{lists} with \empth{sets}.
We instantiate each of three aggregation rounds on top of Handel round.
Because of parallel nature of Handel we have to run multicasting between chained rounds of aggregation in\
order to consistently aggregate.
The resulted construction consists of three Handel rounds and two multicasting rounds in between.
