To overcome the outlined problems, the resulted Spectrum protocol must satisfy the following properties:

\begin{enumerate}
    \item \textbf{Decentralization.} The system should be highly decentralized.
    \item \textbf{Interoperability.} The system should be able to support a large number of heterogeneous blockchains.
    \item \textbf{Openness.} The system should allow anyone to participate in consensus permissionlessly.
    Protocol should be fully open-source and all participants will be encouraged by the incentives system.
    \item \textbf{Consensus Scalability.} The system should be able to operate normally while maintaining\
    sufficiently large consensus groups consisting of hundreds of active validators on each connected blockchain.
    \item \textbf{Operational Scalability.} The system should scale linearly with the number of supported blockchains.
    \item \textbf{Security.} The system should be able to withstand Sybil attacks.
    \item \textbf{Sustainability.} The system should be able to tolerate faults of particular connected blockchains.
    \item \textbf{Upgradability.} The system should allow to add new blockchains into list of supported over time.
\end{enumerate}

To achieve our goals, we will combine the best practices from the approaches, that are already in use\
in the cross-chain interoperability solutions.
To eliminate the existing bottlenecks, we will supplement them with own-developed improvements,\
which we will emphasize and describe in details in the following sections.