So far each member of consensus group had to track changes on all connected chains in order to participate in\
consensus properly.

\textbf{Observation 1:} Events coming from independent systems $S_k$ are not serialized.
Thus, the process of events notarisation can be parallelized.

\textbf{Observation 2:} Outbound transactions on independent systems $S_k$ can be independently signed.

Utilizing those properties we now introduce committee sharding.
We modify protocol in a way such that at each epoch $e$, $K$ distinct committees consisting of nodes equipped with\
functionality unit $F_{S_k}$ relevant to a specific connected chain $S_k$ are selected in a way described in (5.2.2).
All primitives and source of randomness are equal to different committees, the only difference is the $f$ parameter\
in the $\phi(\alpha, f)$ function used in the lottery functional ${\mathcal_{F}}_{L}$.
It is unique for every connected blockchain in order to guaranty a near expected number of members in every committee.
We denote one such committee shard as $V^{e}_{S_k}$, which uniquely maps to $S_k$.
Then, complete mapping of committees to chains at epoch $e$ can be represented as a set of tuples\
committee-chain $\{(V^{e}_{S_k}, S_k)\}$.
Throughout epoch $e$ all events and on-chain transactions on $S_k$ are handled exclusively by $V^{e}_{S_k}$.

Nodes in $V^{e}_{S_k}$ maintain a robust local ledger $L^{local}_k$ of notarized batches of events observed in $S_k$.

\subsubsection{Syncing Shards}

Each committee forms the notarized batches of events from local ledgers $\{L_k\}_{k=1}^{K}$.
All these batches should be synced in a super ledger $L^+$ in order for the system to be able to\
compute cross-chain state transition.
To facilitate this process batches of the notarized events should be broadcast to other committees.
The main actors at this stage are:
\begin{itemize}
    \item \textbf{Local leader}: local committee leader.
    \item \textbf{Relayer}: any protocol participant, who broadcasts notarized batches to the \emph{Local leader}\
    ant to other committees' members.
    Every \emph{Local leader} can be a \emph{Relayer} at the same time.
    \item \textbf{General leader}: one of the \emph{Local leaders} who added a block consisted of collected\
    notarized batches and other internal transactions to the $L^+$.
\end{itemize}

Since there is no separate lottery for the general leadership and any \emph{Local leader} is able to publish his\
block to $L^+$, thus, he can choose from two main strategies:
\begin{itemize}
    \item \textbf{Wait}: malicious strategy where \emph{Local leader} waits for broadcasts from other committees\
    members and don't broadcast his own batch to eliminate competitors for adding a block.
    \item \textbf{Broadcast and wait}: fair strategy where \emph{Local leader} immediately broadcasts his batch,\
    waits for broadcasts from other committees members and honestly adds a block.
\end{itemize}
Thus, there should be a motivation for the individual \emph{Local leader} to choose the fair strategy instead of keeping\
his batch for too long and there also should be a motivation for every committee member to act as a \emph{Relayer}.
This is achieved through the design of the incentive system.

\subsubsection{Incentives}

There are three types of the incentive for the Spectrum protocol participants: ${\{R_b, R_d, R_m\}}$, where $R_b$ is a\
guaranteed reward for adding a notarized batch to the block, $R_d$ is given for broadcasting a batch to the\
\emph{General leader} and $R_m$ is given personally to the \emph{General leader} who finally added the block.
Delivery reward $R_d$ is given if and only if a delivery was made within a predetermined period of time $\Delta t_d$.
From the simple game-theoretic analysis, the following relationships between rewards were\
derived: ${R_b = 2 \cdot R_d, R_m = 3 \cdot R_d}$.
Thus, if ${R_d=0}$ there is no prior strategy for the \emph{Local leaders}, they will or wait for other batches\
either broadcast their batches with equal probability, at the same time all other committee members are motivated to\
act as a \emph{Relayers} to receive an extra reward.
All the rewards except $R_m$ are shared equally between all committees members whose signatures are included in\
the finally added block.

As a result, the syncing Shards flow looks as follows:
\begin{itemize}
    \item After notarization, the committee member holding the notarized batch $b_i$, which contains the local\
    notarization time $t^N_i$ sends it to his \emph{Local leader} and to other known committees members.
    \item All committees members who receive notarized batches from other committees also send it\
    to the \emph{Local leader}.
    \item The \emph{Local leader} collects the received notarized batches.
    \item When waiting time approaches $\Delta t_d$, \emph{Local leader} forms a block from all collected batches\
    ${\{b_i^j\}_{j=1}^{j=K}, K \le N}$ and broadcast it.
    Block contains the set of the notarization times $\{t^{N^j}_i\}_{j=1}^{j=K}$ and block creation time $t^B_s$.
    \item After block is settled, all associated participants receive their rewards according to their roles:\
    \emph{General leader} receives the $R_m$, committee members whose batch is included in the block receives the $R_b$.
    And, in addition, if ${t^B_s - t^N_i^* < \Delta t_d}$, where $t^N_i^*$ is $t^N_i$ time, normalized to $L^+$ time,\
    $i$-th committee members receives the $R_d$ reward.
\end{itemize}
