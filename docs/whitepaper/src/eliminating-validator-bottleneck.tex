So far each member of consensus group had to track changes on all connected chains in order to participate in\
consensus properly.

\textbf{Observation 1:} Events coming from independent systems $S_k$ are not serialized.
Thus, the process of events notarisation can be parallelized.

\textbf{Observation 2:} Outbound transactions on independent systems $S_k$ can be independently signed.

Utilizing those properties we now introduce committee sharding.
We modify protocol in a way such that at each epoch $e$, $K$ distinct committees consisting of nodes equipped with\
functionality unit $F_{S_k}$ relevant to a specific connected chain $S_k$ are selected in a way described in (5.2.2).
All primitives and source of randomness are equal to different committees, the only difference is the $f$ parameter\
in the $\phi(\alpha, f)$ function used in the lottery functional ${\mathcal_{F}}_{L}$.
It is unique for every connected blockchain in order to guaranty a near expected number of members in every committee.
We denote one such committee shard as $V^{e}_{S_k}$, which uniquely maps to $S_k$.
Then, complete mapping of committees to chains at epoch $e$ can be represented as a set of tuples\
committee-chain $\{(V^{e}_{S_k}, S_k)\}$.
Throughout epoch $e$ all events and on-chain transactions on $S_k$ are handled exclusively by $V^{e}_{S_k}$.

Nodes in $V^{e}_{S_k}$ maintain a robust local ledger $L^{local}_k$ of notarized batches of events observed in $S_k$.

\subsubsection{Syncing Shards}

Hence, each committee forms the notarized batches of events from local ledgers $\{L_k\}_{k=1}^{K}$.
All these batches should be synced in a super ledger $L^+$ in order for the system to be able to\
compute cross-chain state transition.
To facilitate this process batches of notarized events are broadcast to other committees.
The main actors at this stage are:
\begin{itemize}
    \item \textbf{Local leader}: committee leader.
    \item \textbf{Relayer}: any protocol participant, who broadcasts notarized batches to other committees' members.
    Every \emph{Local leader} can be a \emph{Relayer} at the same time.
    \item \textbf{General leader}: one of the \emph{Local leaders} who added a block consisted of collected\
    notarized batches to the $L^+$.
\end{itemize}

Since there is no separate lottery for the general leadership and any \emph{Local leader} is able to publish his\
block to $L^+$ he can choose from two main strategies:
\begin{itemize}
    \item \textbf{Wait}: malicious strategy where \emph{Local leader} waits for broadcasts from other committees\
    members and don't broadcast his own batch to eliminate competitors for adding a block.
    \item \textbf{Broadcast and wait}: fair strategy where \emph{Local leader} immediately broadcasts his batch,\
    waits for broadcasts from other committees members and honestly adds a block.
\end{itemize}
Thus, there should be a motivation for individual \emph{Local leader} to choose the fair strategy instead of keeping\
his batch for too long.
This is achieved through the design of the incentive system.

There are three types of incentive: ${\{R_b, R_d, R_m\}}$, where $R_b$ is a guaranteed reward for adding a notarized\
batch to the block, $R_d$ is given for a broadcasting batch to the general leader and $R_m$ is given personally\
to the \emph{General leader} who added the block.
Delivery reward $R_d$ is given to the \emph{Relayer} if and only if a delivery was made within a predetermined\
period of time $\Delta t_d$.
From the simple game-theoretic analysis, the following relationships between rewards were\
derived: ${R_b = 2 \cdot R_d, R_m = 3 \cdot R_d}$.
Thus, if ${R_d=0}$ there is no prior strategy for the \emph{Local leaders}, they will or wait for other batches\
either broadcast their batches with equal probability.
In case when ${R_d>0}$ it is distributed between the \emph{Local leader} and the \emph{Relayer},\
i.e. ${R^l_d = \xi \cdot R_d}$ and  ${R^r_d = (1 - \xi) \cdot R_d}$, where ${\xi \in (0, 1)}$.
While ${\xi \rightarrow 1}$ probability that all \emph{Local leaders} will choose the \emph{Broadcast and wait}\
strategy approaches $1$.

As a result, the syncing Shards flow looks as follows:
\begin{itemize}
    \item Every \emph{Local leader} broadcasts (himself or through an intermediary as a \emph{Relayer}) his batch\
    $b_i$, which contains the local notarization time $t^N_i$ and waits for batches from other \emph{Local leaders}.
    \item When waiting time approaches $\Delta t_d$, \emph{Local leader} forms a block from all collected batches\
    ${\{b_i^j\}_{j=1}^{j=K}, K \le N}$ and add it to $L^+$.
    Block contains the set of the notarization times $\{t^{N^j}_i\}_{j=1}^{j=K}$ and block creation time $t^B_s$.
    \item After block is settled, all associated actors receive their rewards according to their roles:\
    \emph{General leader} receives $R_m$, \emph{Local leaders}, whose batches are in the block receives $R_b$.
    In addition, if ${t^B_s - t^N_i^* < \Delta t_d}$, where $t^N_i^*$ is $t^N_i$ time, normalized to $L^+$ time,\
    $i$-th committee \emph{Local leader} receives $R_d$ reward shared with the \emph{Relayer}.
\end{itemize}

\subsubsection{Forks and integrity}\label{subsubsec:resolving-forks}

Protocol flow implies that any of the local leaders can append their blocks to $L^+$, which leads to forks.
This type of forks is a normal part of the protocol lifecycle, however, total possible number of the normal forks in\
our protocol is much larger than in other blockchains, since there can also be a several local leaders in every\
connected $L_k$ committee.
The chance of occurring a malicious forks produced by adversary is minimized by the lottery design.
In addition, the task for an adversary becomes more difficult by virtue of the interaction between the protocol\
participants during the Syncing Shards process.

The main rules for resolving forks are simple and are performed by the members of all committees when\
validating a proposed blocks:
\begin{itemize}
    \item \textbf{Max valid}: choose the longest appropriate chain given a set of valid chains that are available\
    in the network.
    The depth of any block broadcast by a protocol member during the protocol must exceed the depths of any\
    honestly-generated blocks from slots at least $K$ in the past.
    \item \textbf{Max stake}: if the \emph{Max valid} rule doesn't resolve a slot battle, then the valid chain\
    chooses according to the stake size of the battled leaders, the maximum stake is the winner.
    Stake distribution is picked from the actual blockchain snapshot for the current committee.
\end{itemize}

A large number of the normal forks, however, still significantly affect properties, that maintain the\
integrity of the $L^+$:
\begin{itemize}
    \item \textbf{Latency}: the number of elapsed slots required for a transaction to appear in a block on the $L^+$.
    \item \textbf{Finality}: the number of elapsed slots required for a transaction to become settled and immutable.
\end{itemize}
The Latency of the protocol is good enough due to the short duration of the slots.
Finality is guaranteed after $K_F$ slots, where $K_F$ is a pre-defined protocol parameter.
As a result of the functional features of our protocol, $K_F$ depends on the connected $L_k$ integrity properties.

Most ledgers do not guarantee instant finality of transaction, that means that any (or all) transactions may not\
be applied to corresponding $L_k$ ledgers in the end.
Different blockchains however has different Finality parameters, and time of elapsing $K_F$ should be longer\
than all of them.
Thus, the $K_F$ should be set with a margin and therefore using the number of slots $\Delta Sl$ that have\
passed in the Spectrum network, developers should be able to receive information about the number of blocks that\
have passed in all connected $L_k$ blockchains during this period of time.
The duration of the block in each $L_k$ is different, but the average values are preserved for a certain period of\
time ${\Delta T >> d_s}$, where $d_s$ is the duration of Spectrum's slot.
Thus, after each $\Delta T$ time interval, Spectrum network will update the set of constants:\
${(\{d_{k}\}_{k=1}^{K},\{c_{k}\}_{k=1}^{K})}$, where $d_k$ is a block duration in the $L_k$, $c_k$ is the default\
reliable number of confirmations in the $L_k$, $M$ is the total number of the connected $L_k$.

Using the data above, each Spectrum's $\Delta Sl$ can be associated with the delta of blocks that have passed in\
any connected blockchain: ${\{\lfloor \Delta Sl \cdot d_s \mathbin{/} d_k)\rfloor\}_{k=1}^{K}}$.
When forming transaction, developers can specify a reliability factor $C$.
This factor will be compared with the ratio of the number of blocks passed on the associated $L_k$ blockchain to\
the default reliable number of confirmations $c_k$ of this network:
\begin{equation}
    \theta(k-L_k^{id})\cdot \left\{\frac{1}{c_k} \cdot \left\lfloor \Delta Sl \cdot
    \frac{d_s}{d_k}\right\rfloor\right\}_{k=1}^{K} >= C,\label{eq:equation2}
\end{equation}
where $\theta(x)$ is an indicator function which is 1 at $x = 0$, otherwise 0.
