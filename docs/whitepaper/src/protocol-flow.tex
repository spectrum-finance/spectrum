Let's summarize all of the above and describe the full flow of the Spectrum protocol.
Protocol is running by a set of manually selected opening consensus groups $\{V^k_1\}_{k=1}^K$\
for $K$ connected distributed systems $\{S_k\}_{k=1}^K$.
Each group consists of at least $M_k$ stakeholders interacting with each other and with the ideal\
functionalities $\mathcal{F}_{\text{Init}}$, ${\mathcal{F}}_{\text{VRF}}$, $\mathcal{H}$, ${\mathcal{F}}_{\text{LB}}$,\
$\mathcal{F}_{\text{AggSig}}$, ${\mathcal{F}}_{\text{KES}}$,\
$\mathcal{G}_{\text{ImpLClock}}$ and $\mathcal{G}_{\text{Ledger}}$ over a sequence of $L = E \cdot R$\
slots ${S=\{sl_1,\dots,sl_L\}}$ consisting of $E$ epochs with $R$ slots each.

Functionality ${\mathcal{F}}_{\text{Init}}$~\cite{Badertscher2018} formalizes the procedure of genesis\
block creation and distribution.
Functionality ${\mathcal{F}}_{\text{AggSig}}$ implements the presented in~\ref{subsec:scalable-collective-signature-aggregation} aggregated signature scheme logic.
Functionality $\mathcal{G}_{\text{ImpLClock}}$~\cite{cryptoeprint:2019/838} implements the local\
clock setting and adjusting logic.
Functionality $\mathcal{G}_{\text{Ledger}}$ implements the logic of interaction with the ledger.
Also, each protocol participant maintains at least one functionality unit $\mathcal{F}^k_{\text{ConnSys}}$\
that allows him to interact with the connected $S_k$.

Protocol configuration is represented by publicly known set of constants:\
$R, l_{\text{VRF}}, K_{\text{f}}, K_{\text{g}}, \mathbf{S}_{\text{id}} = \{S_k^{\text{id}}\}_{k=1}^K, \mathbf{f}_{\text{lead}} = \{f^{\text{lead}}_k\}_{k=1}^K, \mathbf{f}_{\text{cons}} = \{f^{\text{cons}}_k\}_{k=1}^K$

\subsubsection{Bootstrapping}\label{subsubsec:bootstrapping}

The system is bootstrapped in a trusted way.
All $M_k$ members of $\{V^k_1\}_{k=1}^K$ committees perform the following procedure:
\begin{enumerate}
    \item On-chain vaults are initialized with an aggregated public key $aPK^k_1$ of the initial committee.

    \item All committee $V^k_1$ members i.e. ${\forall PK_i \in V^k_1}$ must generate the tuple of verification keys\
    ${v_i^{\text{ver}} = (v_i^{\text{vrf}}, v_i^{\text{kes}}, \mathbf{S}_{\text{id}, i})}$, using the ideal\
    functionalities ${\mathcal_{F}}_{\text{VRF}}$ and ${\mathcal{F}}_{\text{KES}}$.
    The tuple also includes a set of ids of the connected distributed systems $\mathbf{S}_{\text{id}, i} \subset \mathbf{S}_{\text{id}}$,\
    the functionalities ${\{\mathcal{F}^k_{\text{ConnSys}}\}_{k=1}^{K'}, K' \leq K}$ to interact with which the participant $PK_i$ is equipped with.
    Verification tuple is committed on-chain in the $\textsf{VerificationRegTx}(v_i^{\text{ver}})$.

    \item Full set of the verification keys tuples\
    ${V_{\text{ver}} = \{(v_i^{\text{vrf}}, v_i^{\text{kes}}, \mathbf{S}_{\text{id}, i})\}_{i=1}^{M}}$\
    with the initial stakes $S = \{s_i\}_{i=1}^{M_k}$ must be stored in the genesis block $B_0$ and\
    acknowledged by all members of the initial consensus group (meaning members of all $\{V^k_1\}_{k=1}^K$ committees).

    \item Functionality ${\mathcal{F}}_{\text{LB}}$, parameterized with the confirmed $V_\text{ver}$\
    is evaluated independently by every\
    participant to sample an initial random seed value $\eta \leftarrow \{0, 1\}^l_{\text{VRF}}$.

    \item Finally, all approved stakeholders should agree on the genesis block\
    ${B_0=\left(V_{\text{ver}}, S, \eta\right)}$.
\end{enumerate}

\subsubsection{Chain Extension}\label{subsubsec:chain-extension}
Once the system is bootstrapped, the Spectrum protocol operates in a normal flow.
Committee $\{V^k_1\}_{k=1}^K$ members adds notarized reports of events observed on external connected systems $\{S_k\}_{k=1}^K$\
into the local ledgers $\{L^{\text{loc}, k}\}_{k=1}^K$.
Blocks with all protocol updates are stored in the common for all participants super ledger $L^+$.
\begin{legal}
    \item Before the epoch $e_n > 2$ begins each protocol participant $PK_i$ must update his state variables:
    \begin{itemize}
        \item[--] Receive new epoch seed $\eta_n$ from the ${\mathcal{F}}_{\text{LB}}$.
        \item [--] Set the leader lottery thresholds for each $k$-th committee he is involved in $\mathbf{T}^{\text{lead}} = \{T_{i, n}^{\text{lead}, k} = \phi_{f^{\text{lead}}_k}(\alpha^{n-2}_{i, \text{k}}\}_{k=1}^{K'}, K' \leq K$,
        where $\alpha^{n-2}_{i, \text{k}}$ is a participant's relative stake relative to $V^k_n$ members\
        according to the state of the blockchain at the end of the epoch $e_{n-2}$.
        \item [--] Set the synchronization lottery threshold $T^{\text{sync}}_{i, n} = 2^{l_{\text{VRF}}}\cdot\phi(\alpha^{n-2}_{i})$,
        where $\alpha^{n-2}_{i}$ is a participant's relative stake relative to all $K$ committees members\
        according to the state of the blockchain at the end of the epoch $e_{n-2}$.
    \end{itemize}

    \item In the epochs first (synchronization) slot each $PK_i$ adjusts his local clocks by $\textsf{shift}_{i, n}$ value calculated according to previously\
    collected synchronization beacons set $\mathbf{b}^{\text{set}}$.

    \item During the epoch all online $V^k_n$ member collects existing chains from $L^+$ and verifying\
    that for every chain, every block, produced up to $K_{\text{f}}$ blocks before contains correct data about the\
    corresponding slot $sl'$ leader $PK'$.
    Each validator must verify that $PK'$ is indeed the winner of the leader lottery for slot $sl'$ as well a valid\
    member of the legitimate committee $V^k_{n'}$.
    All forks must be resolved by the densest chain and largest stake rules in the corresponding priority.

    \item During the epoch, for every slot $sl_j \in [R\cdot n, R\cdot(n+1)]$ every committee $V^k_n$ member $PK_i$ separately evaluates\
    ${\mathcal{F}}_{\test{VRF}}$ with an input ${x_{i, j}^{\text{lead}} = \eta_n || sl_j}$ to receive\
    a slot proof $\pi_{i, j}^{\text{sl}}$ and an associated random value $r_{j}^{\text{sl}}$.

    Then $PK_i$ calculates ${y_{i, j}^{\text{lead}} = \mathcal{H}(r_{j}^{\text{sl}}||\textsf{LEAD}||S_k^{\text{id}})}$ and compares it with the associated\
    threshold $T_{i, n}^{\text{lead}, k}$.
    If $y_{i, j}^{\text{lead}} < T_{i, n}^{\text{lead}, k}$ then the participant is the slot $sl_j$ leader.

    \bigskip
    Leader is allowed to:
    \begin{itemize}
        \item [--] Initiate the notarization round in his local committee $V^k_n$ to add new notarized report into $L^{\text{loc}, k}$.
        \item[--] Propose a new block to be added to the $L^+$.
    \end{itemize}

    \bigskip
    In addition, during the first $R\//6$ slots of the epoch all $PK_i$ checks his right to release a synchronization\
    beacon comparing the pseudo-random value\
    ${y_{i, j}^{\textsf{sync}, n} = \mathcal{H}(r_{j}^{\text{sl}} || \textsf{SYNC})}$\
    with a corresponding threshold $T_{i, n}^{\text{sync}}$.
    If successful then the participant broadcasts a beacon message\
    $b_{i, j}^{\text{sync}} = (v^{\text{vrf}}_i, sl_{i, j}^{\text{loc}}, \pi_{i, j}^{\text{sl}})$.

    \item All committee $V^k_n$ members observe events in their systems $S_k$ and in the $L+$ mempool.
    If $PK_i$ is a slot $sl_j$ leader, then he is able to propose a report $b_j$ of events observed in $S_k$,\
    which should be notarized by other members of the $V^k_n$ using the aggregated signature functionality\
    ${\mathcal{F}}_{\text{AggSig}}$ and then added to the local ledger $L^{\text{loc}, k}$.

    \item Notarized report $b_j^*$ can first be formed by any member of the $V^k_n$.
    The report must be immediately sent to the leader who initiated its notarization and to the\
    members of other committees.
    After the leader receives enough reports he forms a block  $B'$ consisting\
    of all external collected reports and reports from the local $L^{\text{loc}, k}$ that have not yet been added to $L^+$.
    He must include into the block the proof of his leadership $ \pi_{i, j}^{\text{sl}}$,\
    sign the block with ${\mathcal{F}}_{\text{KES}}$\
    and broadcast it to his peers from all committees with the correct signature $\sigma_{\text{KES}}$ included.

    \item After the finality $K_{\text{f}}$ blocks are passed since $B'$ settlement in the $L^+$,\
    all members of all committees that participated in the formation of the block $B'$ can claim their rewards.

\end{legal}

\subsubsection{Epoch Transition}\label{subsubsec:epoch-transition}
\begin{legal}
    \item \textbf{Consensus Group Lottery}.
    \begin{itemize}
        \item [--] At the beginning of the epoch ${e_{n-1} > 2}$ each verified $PK_i$ willing to participate in the
        consensus group lottery for the $V^k_n$ commit his willing in the message $\textsf{VerificationUpdTx(}v_i^{\text{vrf}}, \mathbf{S}_i^{\text{set}}\textsf{)}$
        if he is already verified, or generate verification keys tuple and broadcasts $\textsf{VerificationRegTx}(v_i^{\text{ver}})$.
        \item[--] At end of the epoch $e_{n-1}$ each verified and willing to participate in the consensus lottery $PK_i$
        calculates new consensus lottery thresholds for all committees he wants to be selected in \
        $\{T_{i, n}^{\text{cons}, k} = \phi_{f^{\text{cons}}_k}(\alpha^{n-2}_{i, \text{ver}} \cdot A^{i, n}_{\text{m}}\}_{k=1}^{K'}, K' \leq K$,
        where $\alpha^{n-2}_{i, \text{ver}}$ is a participant's relative stake relative to all verified participants equipped with $\mathcal{F}^k_{\text{ConnSys}}$\
        according to the state of the blockchain at the end of the epoch $e_{n-2}$ and $A^{i, n}_{\text{m}}$ is an activity multiplier.
        \item [--] When every $PK_i$ evaluates ${\mathcal{F}}_{\text{VRF}}$ with input\
        $x_{i, n}^{\textsf{cons}} = \eta_n || e_n $ to receive an epoch proof $\pi_{i, n}^{\text{e}}$.

        \bigskip
        Then for each $S_k^{\text{id}}\in \mathbf{S}_i^{\text{set}}$ calculates\
        the associated random number $y_{i, n}^{\textsf{cons}, k}$ from the proof $\pi_{i, n}^{\text{e}}$, i.e.\
        ${y_{i, j}^{\textsf{cons}, k} = \mathcal{H}(r_{n}^{\text{e}}||\textsf{CONS}||S_k^{\text{id}})}$.
        If $y_{i, j}^{\textsf{cons}, k} < T_{i, n}^{\text{cons}, k}$ then $PK_i$ is a member of $V^k_{n}$ committee.

        \bigskip
        In order to approve the results of the lottery, the participant broadcasts a message with evidence\
        $\textsf{ConsLotteryResTx}(e_n, v_i^{\text{vrf}}, S_k^{\text{id}}, \pi_{i, n}^{\text{e}})$.

    \end{itemize}

    \item \textbf{Committee key aggregation}.
    Once the new committee is selected, nodes in the $V^k_n$ aggregate their individual public keys $PK_i$ into
    a joint one $aPK^k_n$, which is needed to sign the batch applying transactions with the external events:\
    inbound value transfers, outbound value transfers, boxes eliminations.

    \item \textbf{Committee transition}.
    Nodes in the $V^k_{n - 1}$ publish cross-chain message ${m^k_n : (aPK^k_n, \sigma^k_{n-1})}$,\
    where $\sigma^k_{n-1}$ is an aggregated signature such that ${verify: (\sigma^k_{n-1}, aPK^k_{n-1}, m^k_n) = 1}$.
    Finally, vaults are updated such that ${vault^k\{(e_{n-1}, aPK^k_{n-1})\} \coloneqq(e_n, aPK^k_n)}$.
\end{legal}

\subsubsection{Registration}\label{subsubsec:registration}
Any Spectrum stakeholder can register to become a committee member of his local system $S_k$.
To get a chance to be included in the set of validators $V^k_n$ of the epoch $e_n$\
participant $PK_i$ should register in the lottery during the epoch $e_{n-3}$ by publishing his verification tuple\
$(v_i^{\text{vrf}}, v_i^{\text{kes}}, \mathbf{S}_i^{\text{set}})$ into the $L^+$.
Once $K_{\text{f}}$ blocks are added on top of this publication the participant is considered as verified.
Before verification, $PK_i$ must synchronizes with the network by restoring the current chain $C$ from the\
genesis block $B_0$ received from the functionality $\mathcal{F}_{\text{Init}}$.
He also must adjust his local clock based on the synchronization beacons of the current global epoch using\
the functionality $\mathcal{G}_{\text{ImpLClock}}$.
When all synchronization processes are completed, $PK_i$\
is considered a valid participant of the Spectrum protocol.

In the manner described, the Spectrum protocol reaches consensus and implements the cross-chain interoperability.
Our solution is fairly decentralized, fast and scalable, and thus can be used in\
a large number of applications and scenarios.