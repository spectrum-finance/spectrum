\documentclass{article}
\usepackage[utf8]{inputenc}
\usepackage{mathtools}
\usepackage[
    backend=biber,
    style=numeric,
    sorting=none,
    natbib
]{biblatex}
\usepackage{blindtext}
\usepackage{geometry}
\geometry{
    a4paper,
    right=1.5in,
    left=1.5in,
    top=1.5in,
}
\addbibresource{references.bib}

\usepackage[dvipsnames]{xcolor}
\usepackage{color}
\usepackage{csquotes}
\usepackage{textcomp}
\usepackage{amsfonts}
\usepackage{enumitem}
\usepackage{hyperref}
\usepackage[T1]{fontenc}
\usepackage{inconsolata}
\usepackage{amsmath}
\usepackage{amssymb}
\usepackage{algorithm}
\usepackage{algpseudocode}
\usepackage{algorithmicx}

\usepackage{xcolor}
\usepackage{pifont}
\usepackage{listings}
\usepackage{lipsum}
\usepackage{siunitx}
\lstset{
    basicstyle=\color{gray}\ttfamily
}

\hypersetup{
    colorlinks,
    linkcolor={red!30!black},
    citecolor={blue!50!black},
    urlcolor={blue!80!black}
}

\newenvironment{tttabular}[1]
{\ttfamily \begin{tabular}{#1}}
{\end{tabular}}

\newenvironment{protocol}
{
    \begin{center}
        \hrule height.8pt depth0pt \kern2pt
        \renewcommand{\thealgorithm}{}
        \renewcommand{\caption}[2][\relax]{
                {\raggedright\textbf{Protocol} ##2\par}
            \ifx\relax##1\relax
            \addcontentsline{loa}{algorithm}{\protect\numberline{\thealgorithm}##2}
            \else
            \addcontentsline{loa}{algorithm}{\protect\numberline{\thealgorithm}##1}
            \fi
            \kern2pt\hrule\kern2pt
        }
        }{
        \kern2pt\hrule\relax
    \end{center}
}

\newenvironment{functionality}
{
    \begin{center}
        \hrule height.8pt depth0pt \kern2pt
        \renewcommand{\thealgorithm}{}
        \renewcommand{\caption}[2][\relax]{
                {\raggedright\textbf{Functionality} ##2\par}
            \ifx\relax##1\relax
            \addcontentsline{loa}{algorithm}{\protect\numberline{\thealgorithm}##2}
            \else
            \addcontentsline{loa}{algorithm}{\protect\numberline{\thealgorithm}##1}
            \fi
            \kern2pt\hrule\kern2pt
        }
        }{
        \kern2pt\hrule\relax
    \end{center}
}

\newenvironment{algo}
{
    \begin{center}
        \hrule height.8pt depth0pt \kern2pt
        \renewcommand{\thealgorithm}{}
        \renewcommand{\caption}[2][\relax]{
                {\raggedright\textbf{Algorithm} ##2\par}
            \ifx\relax##1\relax
            \addcontentsline{loa}{algorithm}{\protect\numberline{\thealgorithm}##2}
            \else
            \addcontentsline{loa}{algorithm}{\protect\numberline{\thealgorithm}##1}
            \fi
            \kern2pt\hrule\kern2pt
        }
        }{
        \kern2pt\hrule\relax
    \end{center}
}

\newcommand{\lt}{<}
\newcommand{\gt}{>}
\newcommand{\blue}{\color{blue}}

\newlist{legal}{enumerate}{10}
\setlist[legal]{label*=\arabic*.}

\title{Spectrum: Cross-chain interoperability at scale}
\author{Spectrum Labs}
\date{March 2023}

\begin{document}
    \begin{sloppypar}
        \maketitle


        \section{Introduction}\label{sec:introduction}
        Following the success of Bitcoin, many blockchain-based cryptocurrencies have been developed and deployed.
To meet different requirements in various scenarios, a great number of heterogeneous blockchains have emerged.
However, most of the presented blockchain platforms are developed independently, therefore, they are\
designed for their own use cases and incompatible with each other.
Hence, interoperability between blockchains has become one of the key issues\
which prevents blockchain technology from wide adoption.

With fair blockchain interoperability users can potentially conduct transactions across different blockchain networks\
smoothly and without any intermediaries.
This guarantees a reduction in the fragmentation of the crypto ecosystem and opens up new horizons and business models.
Implementation of the blockchain interoperability protocol is challenging since different blockchains have\
different security solutions, consensus algorithms and programming languages.
An inaccurate solution can potentially increase the possibility of attacks and create management challenges\
across different connected networks.

The classic cross-chain interoperability solution is a trusted oracle that registers some event on one blockchain\
and performs the required action on the other.
Centralized oracles provide fast and cheap transactions but lack a key feature -- decentralization.
The liquidity of the protocol built on this approach is custodial which is a centralized approach similar to CeFi when\
users deposit their funds to an exchange's wallet.

Another common approach involves intermediate network consisting of a fixed number of hand-picked oracles to facilitate\
data transfer among multiple blockchains.
The consensus mechanism in such protocols is usually proof-of-authority or proof-of-stake, hence, the wide range\
of potential validators are eliminated due to verification procedures or high collateral and network moderation\
typically carried out by several dozen of rarely alternating nodes.
Moreover, a common practice is to store funds transferred between blockchains on some kind of threshold wallets,\
which are generated by the participants of the intermediate network.
This results in all funds being controlled by a fixed group of oracle operators.
Such a system is also not truly decentralized.

Regarding the application scenarios, one of the most popular in the existing\
blockchain interoperability proposals is an atomic token swap.
However, atomic token swapping protocols~\cite{Miraz2019} are not self-inclusive enough to complete the tasks of\
cross-chain decentralized applications with complex activities than just a token exchanges.
The reason is that the atomic swapping process does not have the ability to destroy a certain amount\
of assets in the source blockchain and re-create the same amount on the target blockchain.
Moreover, this process always requires a counterparty who is willing to exchange tokens~\cite{Schulte2019TowardsBI}.

True blockchain interoperability requires the users and developers have the ability to access information\
from one blockchain inside another without any additional efforts from a third party.
This is a great-efforts task, thus, before achieving a successfully interoperable multi-blockchain system,\
many challenges must be overcome, such as scalability when applying to a large-scale scenario~\cite{Kim2018} and etc.

The motivation of this paper is to describe the Spectrum protocol, which provides an open, truly decentralized,\
secure and scalable cross-chain interoperability solution.
The Spectrum protocol is intended for both end-users and developers, who will be able\
to implement their applications on top of the protocol to widespread\
the use of blockchain technology in various business areas.



        \section{Related Work}\label{sec:related-work}
        Blockchain interoperability is promising, but it still faces various design
challenges.
There have been many systematic researches regarding this issue and many famous authors have discussed\
chain interoperability in general.
The blockchain interoperability in the literature is usually classified into categories.
Buterin~\cite{buterin2016} suggested centralized, sidechains/relays, and hash locking.
Belchior et al.~\cite{belchior2021survey} classified it in cryptocurrency-directed approaches, blockchain engines,\
and blockchain connectors, Wang~\cite{cryptoeprint:2021/537} proposed to group it into\
chain-based interoperability, bridge-based interoperability, and dApp-based interoperability.

\subsection{Existing Interoperability Solutions}\label{subsec:interoperability-categories}
In our work we want to emphasize the benefits of the decentralization in the interoperability mechanism, thus,\
we will not include the systematical-level study of all existing approaches and will briefly discuss\
the classification proposed by~\cite{cryptoeprint:2021/537}.

\subsubsection{Chain-based Interoperability}
The chain-based interoperability aims to public blockchains and uses an atomic swaps as its main mechanism\
to share information across different chains.
Following the classification there are three main types of such interoperability approach:\
hash locking, trusted notary scheme and sidechain.

\textbf{Hash Locking} is an intermediary method, that allows to validate or execute blockchain transactions.
Hashed Time Lock Contracts (HTLCs) were originally developed as an alternative to centralized switching
and can be thought of as a distributed commitment~\cite{Kumar2021}, able to fend off a Byzantine adversary.
It uses a hash time-locked system to lock the transaction~\cite{Pillai2019} which is similar to the concept\
of the cross-chain atomic swap.

From the technical point of view, the hash locking approach has some significant drawbacks,\
for example, it must lock some assets during its opening phase for an established transaction channel,\
thereby creating a race condition and, moreover, the possibility of losing assets if a timeout occurs.

\textbf{Trusted Notary Scheme} usually considers as the simplest way to achieve cross-chain interoperability.
The blockchain notary schemes can provide the functionalities of timed proof of existence,\
whose proof can be used as further proof of ownership~\cite{DIFRANCESCOMAESA202099}.
It doesn't require any additional changes in the underlying blockchains and uses a trusted notary to verify the\
correctness and integrity of information transferred.
A notary can be a stand-alone authority or a group of trusted parties that monitor order books of the connected chains\
and initiating transactions upon the occurrence of some valid events or requests.

Well-known solutions using this technology are, for example, Herdius~\cite{Balazs2017} and Bifrost~\cite{Scheid2019}.
In practice, the most appropriate way to achieve interoperability using a notary scheme is to combine it with\
other methods as it is done in the Interledger~\cite{Thomas2015}, which combines it with sidechain.

\textbf{Sidechain} is the most promising approach in this category.
Sidechain can add new functionalities, namely, security and privacy, to the existing blockchains,
making possible a tokens synchronization and additional data transfer between chains~\cite{Parizi2019}.
The essential feature of the sidechain is that it's design always takes into consideration the structure\
and the consensus of each connected blockchain, but none of the mainchains is aware of the presence of a sidechain.
Moreover, sidechains can be designed in a decentralized manner and have their own consensus protocols.
Sidechain is usually connects to the mainchains with a two-way peg scheme~\cite{SINGH2020102471}, which uses a\
relay routine for a bidirectional hooking.

The main issue is that two-way peg introduces a level of centralization.
Using a two-way pegs introduces a level of centralization, however,\
there are solutions which uses a federated two-way pegs, where single authority is replaced by\
a group of trusted individuals selected in a trustworthy manner.

State-of-the-art sidechain platforms are Loom~\cite{Loom2019}, Liquid~\cite{Nick2020LiquidAB}\
and Poof-of-Authority (PoA) networks~\cite{POA2018}.
There also exist a lot of ongoing projects since this technology is innovative and in demand by the blockchain industry.

Summing it up, a practical way to apply chain-based interoperability methods to current mainstream blockchain\
systems is to combine them together.
Most existing solutions are designed primarily to exchange assets and aiming to improve the transaction processing\
capacity of Bitcoin-dominated tokens.
However, blockchain technology is much wider in its applications, and it's better to focus on transaction\
interoperation between different chains in practical implementations and effectively use all these promising approaches.

\subsubsection{Bridge-based Interoperability}
Bridge-based interoperability aims to create a connection component between homogeneous\
and heterogeneous blockchains.
Solutions in this field are more complex and typically support the extension of smart contracts, which allows\
developers to design and deploy their own logic, thereby expanding the interoperability applications.
Bridge-based interoperability can be implemented in two main forms: trusted Relay and blockchain engine.

\textbf{Trusted Relay} is very native approach where trusted parties share transactions between different blockchains.
Relay schemes replicate block information of the source blockchain via verifiable smart contracts\
within a target blockchain to allow the target blockchain to verify\
the existence of data on the source blockchain without requiring trust in a centralized entity~\cite{buterin2016}.
There exist many developing relay schemes: BTC Relay~\cite{Chow2016}, PeaceRelay~\cite{Luu2019}, etc.
State-of-the art projects are: Hyperledger Cactus~\cite{Hyperledger2020}, Testimonium~\cite{Frauenthaler2020} and\
Tesseract ~\cite{cryptoeprint:2017/1153}.
All this solutions support complex use case and are highly usable and reliable, however,\
still not fully decentralized~\cite{cryptoeprint:2021/537}.

\textbf{Blockchain Engine} is also provides a relay among the connected blockchains.
It is based on a shared infrastructure which support different layers and services, including network, consensus,\
incentive, etc.
Requirements of multi-layer supports is essential, thus, most existing blockchain engine-based solutions are still in\
the stage of proof of concept or under active development.
Most significant projects are: Polkadot~\cite{cryptoeprint:2020/641}, Cosmos~\cite{Kwon2019},
WanChain~\cite{Wanchain}, and ARK~\cite{ARK}.

All bridge-based solutions provide convenience for end-users since they\
do not need to know what happened in the bridge.
In general, trusted relays are much more simple and adopted to handle interoperability.
However, they usually utilize mechanism similar to the notary schemes which also leads\
to a certain degree of centralization.

\subsubsection{dApp-based Interoperability}
Presence of well functioning decentralized applications (dApps) is significant in the blockchain ecosystem, thus,\
dApps should be interoperable as well, this is a goal of dApp-based interoperability.
Each dApp cannot ensure semantic interoperability, and it is essential to develop a minimum semantic that\
must be supported by each application to achieve interoperability among dApps.
dApp-based blockchain interoperability protocols in the literature are classified to: blockchain of blockchains,\
blockchain adapters and blockchain agnostic protocols.

\textbf{Blockchain of Blockchains} is a platform that allows developers to construct a cross-chain dApps, where\
each blockchain functions as an independent one.
It is similar to the sidechain idea, but differs in implementation.
Sidechains typically aimed at atomic swaps among the homogeneous blockchains, where all actions should be coordinated\
by the mainchain.
Blockchain of blockchains solutions typically requires a second layer of blockchain (mainchain)\
to record the activities that happen on each subchain, which can be heterogeneous~\cite{cryptoeprint:2021/537}.
There are several projects where blockchain of blockchains concept is applied for different scenarios:\
Overledger~\cite{Verdian2018}, HyperService~\cite{Liu2019}, SMChain~\cite{cryptoeprint:2019/1401} and etc.

\textbf{Blockchain Adapter} handles the interoperability by providing an interface for the end-users to\
runtime selection, smart contracts, etc.
Most significant project in this category are PleBeuS~\cite{Scheid2020} and smart contracts \emph{move} protocol\
~\cite{Fynn2020}.

\textbf{Blockchain Agnostic Protocol}: refers to a single platform allowing multiple blockchains to co-exist,\
enabling cross-chain or cross-blockchain communication between arbitrarily distributed ledgers.
Blockchain agnosticism provides its end-users various options to pick their optimal blockchain and\
provide the capabilities for cross-chain operations.
Several agnostic-based technologies have been described in the literature: ILPv4~\cite{InterledgerV4},
Gravity~\cite{PupyshevGravity2020}, SuSy~\cite{PupyshevSuSy2020} and etc.
All these solutions are flexible and has great potential, although most of them are focused on the general design\
of the prototype and not grant backward compatibility.

dApp-based blockchain interoperability is very promising, most of the solutions in this category are either\
in early stages of development or lack a practical implementation,\
with criteria to evaluate their effectiveness and efficiency.

\subsubsection{Discussion}

All of the interoperability approaches described above have their strengths and weaknesses.
However, the chain-based interoperability approaches, especially, sidechains, are well-established and\
benefits from extensive research and improvements in design.
Sidechains has two important pros that will help to widespread\
use of blockchain technology in various business areas:
\begin{itemize}
    \item Having their own consensus mechanisms, sidechains can process transactions\
    efficiently and reduce transaction fees for users.
    \item Taking into consideration the structure and the consensus of each connected blockchain,\
    sidechains allow dApps to expand their ecosystem.
\end{itemize}

The main cons of the existing sidechain protocols is a centralization and poor security guaranties of the consensus.
The disadvantages of centralization are obvious:
\begin{itemize}
    \item A system is not sustainable when it depends on a single party.
    \item If the trustee goes down, unfinished swaps can appear frozen halfway.
    \item A malicious trustee can censor transactions.
    \item A malicious trustee can perform a man-in-the-middle attack by sending an inaccurate data.
\end{itemize}
Almost the same deficiencies exist for a semi-centralized protocols,\
where only a few dozen individuals act as validators.
Such \("\)decentralization\("\) is very conditional, as it is difficult to meet the requirements to become a validator,\
furthermore, malicious validators can easily cooperate to successfully attack.

Thus, we come to the conclusion.
A scalable practical implementation of the fairly decentralized system with\
a provably-secure consensus protocol is the main step towards wide practical usage of sidechains and their benefits in\
cross-chain interoperability.



        \section{Goals}\label{sec:goals}
        To overcome the outlined problems, the resulted Spectrum protocol must satisfy the following properties:

\begin{enumerate}
    \item \textbf{Decentralization.} The system should be highly decentralized.
    \item \textbf{Interoperability.} The system should be able to support a large number of heterogeneous blockchains.
    \item \textbf{Openness.} The system should allow anyone to participate in consensus permissionlessly.
    Protocol should be fully open-source and all participants will be encouraged by the incentives system.
    \item \textbf{Consensus Scalability.} The system should be able to operate normally while maintaining\
    sufficiently large consensus groups consisting of hundreds of active validators on each connected blockchain.
    \item \textbf{Operational Scalability.} The system should scale linearly with the number of supported blockchains.
    \item \textbf{Security.} The system should be able to withstand Sybil attacks.
    \item \textbf{Sustainability.} The system should be able to tolerate faults of particular connected blockchains.
    \item \textbf{Upgradability.} The system should allow to add new blockchains into list of supported over time.
\end{enumerate}

To achieve our goals, we will combine the best practices from the approaches, that are already in use\
in the cross-chain interoperability solutions.
To eliminate the existing bottlenecks, we will supplement them with own-developed improvements,\
which we will emphasize and describe in details in the following sections.


        \section{System Model}\label{sec:system-model}
        In this section we will describe the main components and general assumptions which is essential to\
conceptualize and construct the Spectrum protocol.

\subsection{Transaction Ledger}\label{subsec:transaction-ledger.}
We adopt the definition of transaction ledger from~\cite{cryptoeprint:2016/889}.
A protocol $\Pi$ implements a robust transaction ledger, provided that $\Pi$ is divided into blocks that determine\
the order in which transactions are incorporated into the ledger.
Each block in this model is assigned to a specific time slot and the ledger must satisfy the following properties:
\begin{enumerate}
    \item \emph{Persistence.} Once a node of the system proclaims a certain transaction tx as stable, the remaining\
    nodes, if queried, will either report tx in the same position in the ledger or will not report as stable any\
    transaction in conflict to tx.
    Here the notion of stability is a predicate that is parameterized by a security parameter $k$, specifically, a\
    transaction is declared stable if and only if it is in a block that is more than $k$ blocks deep in the ledger.
    \item \emph{Liveness.} If all honest nodes in the system attempt to include a certain transaction then,\
    after time expires corresponding to $u$ slots (called the transaction confirmation time), all nodes, if queried\
    and responding honestly, will report the transaction as stable.
\end{enumerate}

\subsection{Semi-Synchronous Model Preliminaries}\label{subsec:the-semi-synchronous-model-preliminaries.}
We consider the security model in a semi-synchronous setting with simple modifications to account for\
adversarially-controlled message delays and immediate adaptive corruption.

\textbf{Time and Slots.}
In our setting time is divided into discrete units called slots.
The ledger associates one ledger block with each time slot (at most).
Participants are equipped with roughly synchronized clocks.
This will permit them to carry out a distributed protocol intending to collectively assign a block to this current slot.
In general, each slot $sl_r$ is indexed by an integer $r \in \{1, 2, ..\}$, and we assume that the real\
time window that corresponds to each slot has the following two properties:
\begin{itemize}
    \item The current slot is determined by a publicly-known and monotonically increasing function of the current time.
    \item Each participant has access to the current time.
    Any discrepancies between parties' local time are insignificant in comparison with the slot duration.
\end{itemize}

\textbf{Synchrony.}
We consider an untrustworthy network environment that allows for adversarial-controlled message delays and immediate\
adaptive corruption.
Namely, we allow the adversary $A$ to selectively delay any messages sent by an honest party for up to $\Delta \subseteq \mathbb{K}$\
slots and corrupt parties without delay.

\textbf{Random Oracle.}
We assume that a random oracle is available to each node $n \in N$.
The random oracle is designed in such a way that it is able to produce uniformly-distributed pseudo-random numbers\
which correctness must be verifiable for all participants of the protocol.

\textbf{Security Model.}\label{subsec:security-model.}
The system is composed of a set of nodes $N$ and each node $n \in N$:
\begin{itemize}
    \item Is associated with a unique wallet holding a stake of tokens $s_n$.
    \item Able to generate key-pairs ${(PK, SK)}$ without trusted public key infrastructure.
    \item Is able to sign messages ${sign: (SK, m) \rightarrow \sigma}$.
    \item Is able to verify signatures ${verify: (\sigma, PK, m) \rightarrow 0 | 1}$.
    \item Has access to random oracle functionality.
    \item Has access to key evolving signature functionality.
\end{itemize}

At any time $t$ a subset ${V \subseteq N}$ of nodes can be controlled by an adversary and are considered faulty.
Byzantine nodes can divert from the protocol and collude to attack the system while the remaining honest nodes follow\
the protocol.
We assume that the total stake of all faulty nodes is less than 1/3 of the total stake of all nodes.

\subsection{External Systems}\label{subsec:external-systems.}
We also assume multiple independent distributed systems ${S_1, \dots, S_K}$ with underlying ledgers ${L_1, \dots, L_K}$\
as defined in~\cite{cryptoeprint:2019/1128}.
For each ledger ${L_k, k \in K}$ there is a process $P_k$ that can influence the state evolution\
of the underlying ledger $L_k$ by committing a transaction $TX_k$ into it.
We extend the model defined in~\cite{cryptoeprint:2019/1128} by assuming that all ledgers allow for execution of\
simple predicates upon validation of transactions: ${verify: C \rightarrow 0 | 1}$, where $C$ is\
a \emph{context} that contains description of state the transaction interacts with.
There is also a function ${desc: TX_k \rightarrow DESC^{TX_k}}$ that maps transaction $TX_k$ to\
some \emph{description}, e.g.\ specifying the transaction value, recipient address, etc.
For each $S_k$ there is a corresponding functionality unit $F_{S_k}$ that allows any node equipped with the unit\
to interact with $S_k$.
Each node $n \in N$ is equipped with at least one such functionality unit and at most $K$ functionality units.



        \section{System Design}\label{sec:system-design}
        This section presents Spectrum protocol design starting from a naive approach based on PBFT and gradually\
addressing the challenges.

\subsection{Strawman Design: PBFTNetwork}\label{subsec:strawman-design}

For simplicity we begin with a notarization protocol based on PBFT, then iteratively refine it into Spectrum.

PBFTNetwork assumes that a group of ${n = 3f + 1}$ trusted nodes has been pre-selected upfront and fixed and at most\
$f$ of these nodes are byzantine.
At any given time one of these nodes is the \emph{leader}, who observes events on connected blockchains,
batch them and initiate round of notarization within the consensus group.
Remaining members of the consensus group verify the proposed batches by checking the presence of updates on\
corresponding blockchains.
Upon successful verification each node signs the batch with its secret key and sends the signature to the leader.

Under simplifying assumptions that at most $f$ nodes are byzantine the PBFTNetwork guarantees livness and safety.
However, the assumption of a fixed trusted committee is not realistic for open decentralized systems.
Moreover, as PBFT consensus members authenticate each other via non-transferable symmetric-key MACs, each consensus
member has to communicate with others directly, what results in $O(n^2)$ communication complexity.
Quadratic communication complexity imposes a hard limit on scalability of the system.
Such a design also scales poorly in terms of adding support for more chains.
The workload of each validator grows linearly with each added chain.

In the subsequent sections we address these limitations in four steps:
\begin{enumerate}
    \item \textbf{Opening consensus group and leaders.} We introduce a lottery-based mechanism for selecting consensus\
    group and leaders dynamically.
    \item \textbf{Replacing MACs by Digital Signatures.} We replace MACs by digital signatures to make authentication\
    transferable and thus opening the door for sparser communication patterns that can help to reduce\
    the communication complexity.
    \item \textbf{Scalable Collective Signature Aggregation.} We utilize Byzantine-tolerant aggregation\
    protocol that allows for quick aggregation of cryptographic signatures to reduce communication complexity\
    to $O(\log n)$.
    \item \textbf{Eliminating Validator Bottleneck.} We shard consensus groups into units by the type of chain\
    each node is able to handle.
\end{enumerate}

\subsection{Opening Consensus Group}\label{subsec:opening-consensus-group-and-leaders}
Spectrum is an open-membership protocol, so PBFTNetwork's assumption on a closed consensus group is not valid.
Sybil attacks can break any protocol with security thresholds and an appropriate dynamic selection of\
the consensus group becomes crucial for preserving network's liveness and safety.
Election of consensus group members should be performed in a random and trustless way to ensure that\
a sufficient fraction (at most $f$ out of ${3 f + 1}$) of members are honest.

Similar selection mechanics is required in most blockchain protocols.
Bitcoin~\cite{nakamoto2009bitcoin} and many its successors are using Proof-of-Work (PoW) consensus,\
which, in essence, is a robust mechanism that facilitates randomized selection of a leader who is\
eligible to produce a new block.
Later, PoW approach was adapted into a Proof-of-Membership mechanism~\cite{kokoriskogias2016enhancing}.\
This mechanism allows once in a while to select a new consensus group\
which then executes the PBFT consensus protocol.

A primary consideration regarding PoW-based consensus mechanisms is\
the amount of energy required to operate such systems.
A natural alternative to PoW is a mechanism based on the concept of Proof-of-Stake (PoS)~\cite{King2012PPCoinPC}.
Rather than investing computational resources in order to participate in the leader selection process,\
participants of a PoS system instead run a process that randomly selects one of them proportionally to the stake.
Pure PoS mechanism to solve the PBFT problem was firstly used in~\cite{cryptoeprint:2017/454} to select both consensus\
group members and PBFT rounds leaders and to introduce randomness into this process,\
a verifiable Random Function (VRF) has been applied.

\subsubsection{Verifiable Random Function}

A Verifiable Random Function (VRF)~\cite{Micali1999} is a reliable way to introduce randomness into a protocol.
By definition, a function $\mathcal{F}$ can be attributed to the VRF family if the following methods are defined\
for the $\mathcal{F}$:
\begin{itemize}
    \item[--] Gen: ${Gen(1^l) \rightarrow (PK, SK)}$, where $PK$ is the public key and $SK$ is the secret key.
    \item[--] Prove: ${Eval(x, SK) \rightarrow \pi}$, where $x$ is an input and $\pi \vcentcolon= \Pi(x, SK)$ is\
    the proof, associated with $x$ and mixed with a random value, sampled from $\{0,1\}^{l_{\text{VRF}}}$.
    \item[--] Verify: ${Verify(x, \pi, PK) \rightarrow 0 | 1}$, where the output is $1$ if\
    and only if ${\pi \equiv \Pi(x, SK)}$.
\end{itemize}

The most secure implementations of VRF nowadays are Elliptic Curve Verifiable Random Functions (ECVRFs).
Basically, ECVRF is a cryptographic-based VRF that satisfies the uniqueness, collision resistance,\
and full pseudorandomness properties~\cite{cryptoeprint:2014/905}.
The security of ECVRF follows from the decisional Diffie-Hellman assumption in the random oracle model, thus\
ECVRF is a good source of randomness for a blockchain protocol.
Using ECVRF is also cheap and fast, since single ECVRF evaluation is approximately 100 microseconds on\
x86-64 for a specific curves used in hash functions.
Moreover, there is a great UC-extension for batch verification proposed by~\cite{cryptoeprint:2022/1045}\
which make it even faster by reducing the number of evaluations.

\subsubsection{Lottery}
Our lottery mechanism is based on ECVRF as a source of randomness and is generally inspired\
by Ouroboros Praos~\cite{cryptoeprint:2017/573} and Algorand~\cite{cryptoeprint:2017/454}.
The lottery mechanism in general allows the protocol assign a specific \emph{role} to a participant,\
while the validity of the participant's role can be verified using only publicly available data.

The main assigning logic is as follows:
\begin{legal}
    \item Participant calculates a certain threshold value $T$ according to predefined rules and\
    using only publicly available data for the calculation.
    \item Participant evaluates VRF function and calculates a random number $y$ using the VRF's proof $\pi$.
    \item If ${y < T}$ then the participant is considered valid for the respective role.
\end{legal}

To be more precise, let's clarify that in our setting a threshold value $T$ is calculated according\
to the formula ${T = 2^{l_{\text{VRF}}}\cdot \phi_{f}(\alpha, f)}$ where\
${\alpha=s/\\\sum_{i=1}^{M} s_i}$ is a relative stake.
Consequently, the probability of winning is calculated as ${p(\alpha, f) = 1-(1-f)^{\alpha}}$.
Thus, the winning probability depends on the participant's relative stake and is adjusted by the free parameter $f$.
This is where the PoS concept comes into play: the bigger the stake, the higher the chance of winning the lottery.

The lottery mechanism is fast, secure, and adaptive, since the involved pre-defined parameters\
can be changed via the voting process.
Moreover, the same primitives can be used to achieve different goals and we will utilize the lottery mechanism\
in several aspects of our protocol.

\textbf{Consensus Group Lottery}.
In the current section, we are considering a lottery mechanism application for \emph{dynamic consensus group selection}.
The Spectrum protocol initially is running by the manually selected opening consensus group $\{PK_i\}_{i=1}^M$\
of the predefined size $M$.
Stakeholders interact with each other and with locally installed ideal functionalities\
$\mathcal{F}_{\text{VRF}}$ and $\mathcal{F}_{\text{LB}}$ over a sequence of $L = E \cdot R$ slots\
${S=\{sl_1,\dots,sl_L\}}$ consisting of $E$ epochs with $R$ slots each.

Let's clarify what the mentioned above pre-defined primitives are needed for.
The ideal Verifiable Random Function functionality ${\mathcal{F}}_{\text{VRF}}$ we use here is similar to the extended VRF functionality\
introduced by Christian Badertscher et al.~\cite{cryptoeprint:2022/1045}:

Ideal Leaky Beacon functionality $\mathcal{F}_{\text{LB}}$ is used to sample an epoch random seed from the\
blockchain and is defined as follows:
\begin{functionality}
    \caption{${\mathcal{F}_{\text{LB}}(e_n, C_{\text{loc}})}$}
    \begin{algorithmic}
        \State \lstinline|// New epoch random seed is sampled once per epoch.|
        \State \lstinline|// C_loc is the local chain of the validator.|
        \If{${e_n < 2}$}
            \State \Return ${\textsf{false}}$
        \EndIf
        \For {each ${B_k \in C_{\text{loc}} \ |\ (B_k\textsf{.get(}e\textsf{)} \leq e_{n - 1})\ \wedge (\forall B_k\textsf{.get(}sl\textsf{)} \in R \cdot (n - 1) \cdot 2 /\\ 3)}$}
            \State \lstinline|// Every block B_k in the C_loc was produced by i'-th leader|
            \State \lstinline|// during j'-th slot, i.e. k = (i', j').|
            \State ${\pi^{\text{sl}} \leftarrow B\textsf{.get(}\pi^{\text{sl}}\textsf{)}}$
            \State ${r^{\text{sl}} \leftarrow \textsf{extract\_random(}\pi^{\text{sl}}\textsf{)}}$
            \State $y^{\text{rand}} \leftarrow \mathcal{H}(r^{\text{sl}} || \textsf{RAND})$
            \State ${\eta_n \leftarrow \mathcal{H}(\eta_{n - 1} || e_n || y^{\text{rand}})}$
        \EndFor
        \State \Return $\eta_n$
    \end{algorithmic}
\end{functionality}
An extended formal analysis of the security guaranties of the ${\mathcal{F}}_{\text{LB}}$ can be found in the original Ouroboros Praos\
paper~\cite{cryptoeprint:2017/573}.

Consensus group is constantly rotated each epoch ${e_n \gt 2}$.
Any verified protocol participant $PK_i$ can try to become a temporal member of the consensus group.
Participant is verified if his verification key tuple $v_i^{\text{ver}}$ is published in the blockchain during\
the epoch $e_{j-2}$ in the special $\textsf{VerificationRegTx}(v_i^{\text{ver}})$.
The consensus group lottery flow is as follows:
\begin{enumerate}
    \item At the end of the epoch ${e_n \gt 2}$ every verified participant $PK_i$ requests a\
    new epoch seed $\eta_n$ from the ${\mathcal{F}}_{\text{LB}}$.
    \item New consensus lottery threshold $T^{\text{cons}} = \phi_{f^{\text{cons}}}(\alpha^{n-2}_i)$\
    is calculated by every $PK_i$\ using stake distribution (to get the relative stake $\alpha^{n - 2}_{i}$)\
    from the blockchain state at the last block of the epoch $e_{n - 2}$.
    Free parameter $f^{\text{cons}}$ of the associated function $\phi$ is ${f^{\text{cons}} = M_n /\/ N_n}$,\
    where $M_n$ is a pre-defined number of new consensus group members to select at epoch $e_n$\
    and $N_n$ is the total number of verified stakeholders.
    \item When every $PK_i$ evaluates ${\mathcal{F}}_{\text{VRF}}$ with input\
    $x^{\text{cons}} = \eta_n || e_n $ and calculates the associated random number $y_{i, n}^{\text{cons}}$ from the received proof $\pi_{i, n}^{\text{e}}$, i.e.\
    ${y_{i, n}^{\text{cons}} = \mathcal{H}(r_{i, n}^{\text{e}}||\textsf{CONS})}$, where $r_{i, n}^{\text{e}}$\
    is a random number extracted from the proof and $\textsf{CONS}$ is an arbitrary pre-defined constant.
    \item To reveal the result of the consensus group lottery $PK_i$ compares value $y_{i, n}^{\text{cons}}$\
    with the threshold $T_{i, n}^{\text{cons}}$.
    If ${y_{i, n}^{\text{cons}} < T_{i, n}^{\text{cons}}}$\
    then the participant is a legal member of new consensus group which will be active in the epoch $e_{n+2}$.
    \item Finally, to declare his right to participate in the new consensus group, participant $PK_i$\
    includes an associated proof $\pi_{i, n}^{\text{e}}$ into the\
    $\textsf{ConsLotteryResTx}(e_n, v_i^{\text{vrf}}, \pi_{i, n}^{\text{e}})$ and adds it into the main chain.
\end{enumerate}
Note, that the members of the consensus group should be known ahead of time for the synchronization.
Therefore, in order to participate in the $e_n$ consensus lottery already verified participant must\
publish $\textsf{VerificationUpdTx}$ message at the epoch $e_{n-2}$.
Public disclosure of the future consensus group doesn't give much advantage to an adversary\
since there are hundreds of consensus members in every epoch and denial of service attacks are difficult to succeed.
At the same time any grinding attacks are limited because an adversary can't arbitrarily control $\eta_n$ values.

The main task of the validators set elected via the consensus group lottery is to observe and notarize\
events using a digital signature aggregation mechanism which we will introduce in the next sections.


\subsection{Replacing MACs by Digital Signatures}\label{subsec:replacing-macs-by-digital-signatures}

todo

\subsection{Scalable Collective Signature Aggregation}\label{subsec:scalable-collective-signature-aggregation}

todo

\subsection{Spectrum Ledger}\label{subsec:ledger}
Spectrum network is complex since it works with multiple connected blockchains.
In order to accurately describe and verify all the necessary properties of the Spectrum ledger, we should agree on the\
semantics and introduce data types that satisfy some validity conditions which we will describe in this section.


todo

\subsubsection{Forks}\label{subsec:resolving-forks}

Protocol flow implies that there can be a several local leaders in every connected $L_k$ committee,\
which leads to forks.
This type of fork is a normal part of the protocol lifecycle, however, total possible number of the normal forks in\
our protocol is much larger than in other blockchains, since any of the local leaders can append their blocks to $L^+$.
The chance of occurring a malicious forks produced by adversary is minimized due to the lottery\
and the incentive mechanism design.
In addition, the task for an adversary becomes more difficult by virtue of the interaction between the protocol\
participants during the syncing shards process.

For the above reasons, the main rules for resolving forks are simple and are\
performed by members of all committees when validating a proposed blocks:
\begin{enumerate}
    \item \textit{Max valid}: choose the longest appropriate chain given a set of valid chains that are available\
    in the network.
    \item \textit{Max stake}: if the max valid rule doesn't resolve a slot battle, then the valid chain\
    chooses according to the real stake size of the battled leaders, the maximum stake is the winner.
    Stake distribution is picked from the actual blockchain snapshot for the current committee.
\end{enumerate}

However, a large number fo forks, still significantly affect properties, that maintain the integrity of the $L^+$:
\begin{enumerate}
    \item \textit{Latency}: the number of elapsed slots required for a transaction to appear in a block on the $L^+$.
    \item \textit{Finality}: the number of elapsed slots required for a transaction to become settled and immutable.
\end{enumerate}
The latency of the protocol is good enough due to the short duration of the slots, while the finality,\
as a result of the functional features of our protocol, depends on the connected $c_k$ integrity properties.

Most ledgers do not guarantee instant finality of transaction, that means that any (or all) transactions may not\
be applied to the corresponding $c_k$ ledgers in the end.
Different blockchains has different finality parameters, and finality time of $K_f$ should be longer\
than all of them.
Thus, the $K_f$ should be set with a margin and, therefore, using the number of slots $\Delta sl$ that have\
passed in the Spectrum network, developers should be able to receive information about the number of blocks that\
have passed in all connected blockchain during this period of time.
The duration of the block in each $c_k$ is different, but the average values are preserved for a certain period of\
time ${\Delta T >> d_s}$, where $d_s$ is the duration of Spectrum's slot.
Thus, after each $\Delta T$ time interval, Spectrum network will update the set of constants:\
${(\{d_{k}\}_{k=0}^{K},\{r_k\}_{k=0}^{K})}$, where $d_k$ is a block duration in the $L_k$, $r_k$ is the default\
reliable number of confirmations in the $c_k$ and $K$ is the total number of the connected $c_k$.

Using the data above, each Spectrum's $\Delta sl$ can be associated with the delta of blocks that have passed in\
any connected blockchain: ${\{\lfloor \Delta sl \cdot d_s \mathbin{/} d_k)\rfloor\}_{k=0}^{K}}$.
When forming transaction, developers can specify a reliability factor $C$.
This factor will be compared with the ratio of the number of blocks passed on the associated $L_k$ blockchain to\
the default reliable number of confirmations $r_k$ of this network:
\begin{align*}
    \theta(k-c_k^{id})\cdot \left\{\frac{1}{r_k} \cdot \left\lfloor \Delta sl \cdot
    \frac{d_s}{d_k}\right\rfloor\right\}_{k=0}^{K} >= C,
\end{align*}
where $\theta(x)$ is an indicator function which is 1 at $x = 0$, otherwise 0.

\subsubsection{Handling Rollbacks}\label{subsec:rollbacks}
todo

\subsubsection{State Transition System}
todo

\subsection{Eliminating Validator Bottleneck}\label{subsec:eliminating-validator-bottleneck}
So far, each member of the consensus group had to track changes on all connected chains in order to participate in\
consensus properly.
However, this approach reduces the number of possible consensus participants and limits the scalability of the system.
Therefore, for the optimal design of our consensus protocol, we will use the following observations:
\begin{itemize}
    \item[]\textbf{Observation 1:} Events coming from independent systems $S_k$ are not serialized.
    \item[] \textbf{Observation 2:} Outbound transactions on independent systems $S_k$ can be independently signed.
\end{itemize}

Utilizing those properties, we now introduce committee sharding.
We modify protocol in a way such that at each epoch $e_n$, $K$ distinct committees consisting of nodes equipped with\
functionality unit $F_{L_k}$ relevant to a specific connected system $S_k$ are selected via the consensus group lottery.
All primitives using in the lottery are equal for different committees, however, lotteries are independent.

We denote one such committee shard as $V_n^k$, which uniquely maps to $S_k$.
Then, complete mapping of committees to chains at epoch $e_n$ can be represented as a set of tuples\
committee-chain $\{(V_n^k, S_k)\}$.
Throughout epoch $e_n$ all events and on-chain transactions in $S_k$ are handled exclusively by $V_n^k$.
Nodes in $V_n^k$ maintain a robust local ledger $L^{local}_k$ of notarized batches of events observed in $S_k$.

\subsubsection{Syncing Shards}

Each committee $V_n^k$ forms the notarized batches of events and adds them into their local ledgers $L_k$.
All these batches should be periodically synced and added to a block of the main super ledger $L^+$\
in order the system to be able to compute a cross-chain state transition.
To facilitate this process, batches of the notarized events should be broadcast to other committees.
The main actors at this stage are:
\begin{enumerate}
    \item \emph{Local leader}: local committee leader.
    \item \emph{Relayer}: any protocol participant that broadcasts notarized batches to the local leader\
    and to other committees' members.
    Every local leader can be a relayer at the same time.
    \item \emph{General leader}: one of the local leaders who added a block consisted of collected\
    notarized batches and other internal transactions to the $L^+$.
\end{enumerate}

There is no any separate lottery for the general leadership and any local leader is able to publish his\
block to $L^+$, thus, he can choose from two main strategies:
\begin{enumerate}
    \item \emph{Wait}: malicious strategy where local leader waits for broadcasts from other committees\
    members and don't broadcast his own batch to eliminate competitors for adding a block.
    \item \emph{Broadcast and wait}: fair strategy where local leader immediately broadcasts his batch,\
    waits for broadcasts from other committees' and then competes honestly for adding a block.
\end{enumerate}
There should be a motivation for an individual local leader to choose the fair strategy instead of keeping\
his batch for too long and there also should be a motivation for every committee member to act as a relayer.
This is achieved through the design of the incentive system.

\subsubsection{Incentives}

There are three types of the incentive for the Spectrum protocol participants: ${\{R_b, R_d, R_m\}}$, where $R_b$ is a\
guaranteed reward for adding a notarized batch to the block, $R_d$ is given for broadcasting a batch to the\
general leader and $R_m$ is given personally to the general leader who will finally add the block.
Delivery reward $R_d$ is given if and only if a delivery was made within a predetermined period of time $\Delta t_d$.

Reward amounts are initially configured in such a ratio that if ${R_d=0}$ there is no prior strategy for\
local leaders, they will either wait for other batches or broadcast their batches with equal probability.
At the same time, all other committee members are motivated to act as a relayers to receive an extra reward,
since the notarized batch can be firstly generated by any member of the committee.
All the rewards except $R_m$ are shared equally between all committees members whose signatures are included in\
the finally added block.

As a result, the syncing shards flow looks as follows:
\begin{enumerate}
    \item After notarization, a committee member holding the notarized batch which contains the local\
    notarization time, sends it to his local leader and to other known committees members.
    \item All committees members who receive notarized batches from other committees also send them\
    to the local leader.
    \item The local leader collects the received notarized batches.
    \item When waiting time approaches $\Delta t_d$, the local leader forms and broadcasts a block consisting\
    of all external collected batches and batches from the local $L_k$ that have not yet been added to $L^+$.
    \item After block is reliably settled in the $L^+$, all associated participants can claim their rewards.
\end{enumerate}

We also introduce another type of authority incentive that increases the chances of participants\
to be selected in the consensus group lottery.
When calculating the steak distribution which is needed to parametrise the lottery function,\
all stakes are weighted depending on the actions of their holders in the previous epoch,\
i.e. ${s_i = A_m \cdot s_i^{real}}$, where $A_m$ is an authority multiplier.
If some authority was a member of the previous committee and participated in adding of 2/3 of the blocks produced in\
the considered period of time, then his actual stake ${s_i^{real}}$ is multiplied by ${A_m = 2}$.
Multiplier $A_m$ decreases linearly to 0, which is the case where member was passive during the entire epoch.

With this mechanism, we solve the following problems:
\begin{itemize}
    \item Members are motivated to be focused on cooperation with other committees\
    so that their participation is reflected in each block added in the $L^+$.
    \item Inactive and dishonest members are automatically excluded from the next epoch committee.
    \item Participants are motivated to stay active throughout the entire epoch so that their chances of being\
    selected in the committee don't decrease due to an authority multiplier ${A_m < 1}$, otherwise,\
    in order to even the odds with new lottery participants, they will either have to increase\
    their real stake or skip the lottery until the next one.
\end{itemize}

\subsubsection{Forks and Integrity}\label{subsubsec:resolving-forks}

Protocol flow implies that there can be a several local leaders\
in every connected $S_k$ committee, which leads to forks.
This type of fork is a normal part of the protocol lifecycle, however, total possible number of the normal forks in\
our protocol is greater than in other blockchains, since any of the local leaders can append their blocks to $L^+$.
The chance of occurring a malicious forks produced by an adversary is minimized due to the lottery\
and the incentive mechanism design.
In addition, the task for an adversary becomes more difficult by virtue of the interaction between the protocol\
participants during the syncing shards process.

For the above reasons, the main rules for resolving forks are simple and are\
followed by members of all committees when validating a proposed blocks:
\begin{enumerate}
    \item \textit{Max valid}: choose the longest appropriate chain given a set of valid chains that are available\
    in the network.
    \item \textit{Max stake}: if the max valid rule doesn't resolve a slot battle, then the valid chain\
    chooses according to the real stake size of the battled leaders, the maximum stake is the winner.
    Stake distribution is picked from the actual blockchain snapshot for the current committee.
\end{enumerate}

However, a large number fo forks still significantly affect properties that maintain the integrity of the $L^+$:
\begin{enumerate}
    \item \textit{Latency}: the number of elapsed slots required for a transaction to appear in a block on the $L^+$.
    \item \textit{Finality}: the number of elapsed slots required for a transaction to become settled and immutable.
\end{enumerate}
The latency of the protocol is good enough due to the short duration of the slots, while the finality,\
as a result of the functional features of our protocol, depends on the connected $S_k$ integrity properties.

Most ledgers do not guarantee instant finality of transaction, that means that any (or all) transactions may not\
be applied to the corresponding $S_k$ ledgers in the end.
Different blockchains has different finality parameters, and the Spectrum finality time $U_f$ should be greater\
than all of them.
Thus, the $U_f$ should be set with a margin and, therefore, using the number of slots $\Delta sl$ that have\
passed in the Spectrum network, developers should be able to receive information about the number of blocks that\
have passed in any connected blockchain during this period of time.
The duration of block creation in each $S_k$ is different, but the average values are preserved for a certain period of\
time ${\Delta T >> d_s}$, where $d_s$ is the duration of Spectrum's slot.
Thus, after each $\Delta T$ time interval, Spectrum network will update the set of constants:\
${(\{d_k, U_k\}_{r=1}^{K})}$, where $d_k$ is a block duration in the $S_k$ and $r_k$ is the default\
reliable number of confirmations in the $S_k$ and $K$ is the total number of the connected systems.

Using the data above, each Spectrum's $\Delta sl$ can be associated with the delta of blocks that have passed in\
any connected blockchain: ${\{\lfloor \Delta sl \cdot d_s \mathbin{/} d_k)\rfloor\}_{k=1}^{K}}$.
When forming transaction, developers can specify a reliability factor $r^*$.
This factor will be compared with the ratio of the number of blocks passed on the associated $S_k$ to\
the default reliable number of confirmations $r_k$ of this system.

The ability to access this information is important for tracking the status of value carrying units in\
the Spectrum's global state.
The aspects of the implementation of our ledger is described in detail in the next section.




\subsection{Protocol Flow}\label{subsec:protocol-flow}
let's summarize all of the above and describe the full flow of the Spectrum protocol.
Protocol is running by manually selected opening consensus group $V_0$.
Stakeholders interact with each other and with the ideal functionalities ${\mathcal_{F}}_{RLB}$,\
${\mathcal_{F}}_{VRF}$, ${\mathcal_{F}}_{KES}$, ${\mathcal_{F}}_{DSIG}$ over a sequence of $L = E \cdot R$ slots\
${S=\{sl_1,...,sl_L\}}$ consisting of $E$ epochs with $R$ slots each.

\subsubsection{Bootstrapping}\label{subsubsec:bootstrapping}

The system is bootstrapped in a trusted way.
A manually picked set of validators $V_0$ of the predefined size $M$ is assigned to the first epoch $e_0$.
\begin{enumerate}
    \item On-chain vaults are initialized with an aggregated public key $aPK_0$ of the initial committee.

    \item All consensus group members i.e. $\forall PK_i \in V_0$ should generate the tuple of verification keys\
    ${(v_i^{vrf}, v_i^{kes}, v_i^{dsig})}$, using the ideal functionalities ${\mathcal_{F}}_{VRF}$,\
    ${\mathcal_{F}}_{KES}$, ${\mathcal_{F}}_{DSIG}$ instances, running on their machines.

    \item Then, to claim an initial stakes $\{s_i\}_{i=0}^M$ every protocol participant sends a request\
    ${(\textbf{ver\_keys}, sid, PK_i, v_i^{vrf}, v_i^{kes}, v_i^{dsig})}$ to the ${\mathcal_{F}}_{RLB}$,\
    which saves the key tuple ${(PK_i, v_i^{vrf}, v_i^{kes}, v_i^{dsig})}$.

    \item Full set of the verification\
    keys tuples ${V_{init} = \{(PK_i, v_i^{vrf}, v_i^{kes}, v_i^{dsig})\}_{i=0}^M}$\
    should be stored in the blockchain and acknowledged by all members of the initial consensus group.

    \item ${\mathcal_{F}}_{RLB}$, parameterized with confirmed $V_{init}$ is evaluated independently by every\
    participant to sample an initial random seed value $\eta \leftarrow \{0, 1\}^\lambda$.

    \item Finally, all approved stakeholders should agree on the genesis block\
    ${B_0=\left(\{(PK_i, v_i^{vrf}, v_i^{kes}, v_i^{dsig}, s_i)\}_{i=0}^M, \eta\right)}$.
\end{enumerate}

\subsubsection{Normal Flow}\label{subsubsec:normal-flow}
Once the system is bootstrapped, protocol is running in a normal flow.
\begin{enumerate}[listparindent=0.5cm,
    align=left]
    \item \textbf{Registration}.
    Any Spectrum stakeholder can register to become a committee member.
    To get a chance to be included in the set of validators $V_n$ in the epoch $e_n$ participant $PK_i$ should register\
    in the lottery during the epoch $e_{n-2}$ by publishing his verification tuples\
    ${(v_i^{vrf}, v_i^{kes}, v_i^{dsig})}$.

    \item \textbf{Consensus Lottery}.
    At the end of the epoch ${e_j \geqslant 1}$ every verified $PK_i$ sends\
    ${(\textbf{epochrnd\_req}, sid, PK_i, e_j)}$ to ${\mathcal_{F}}_{RLB}$ and receives\
    $({\textbf{epochrnd}, sid, \eta_j)}$.
    When every $PK_i$ evaluates ${\mathcal_{F}}_{VRF}$ with a query\
    ${(\textbf{EvalProve}, sid, \eta_j || e_j || \textbf{test*})}$\
    and pass the received random $y$ value to the ${\mathcal_{F}}_{L}$.
    If successful, then publish $y$ and the associated proof to form an approved consensus\
    group members table for the epoch ${e_{j + 1}}$.

    \item \textbf{Committee key aggregation}.
    Once new committee is selected, nodes in $V_n$ aggregate their individual public keys $\{PK_i\}$ into
    a joint one $aPK_n$.

    \item \textbf{Committee transition}.
    Nodes in $V_{n-1}$ publish cross-chain message ${M_n : (aPK_n, \sigma_{n-1})}$ , where $aPK_n$ is
    an aggregated public key of the new committee $V_n$ , $\sigma_{n-1}$ is an aggregated signature of
    $M_n$ such that ${Verify(\sigma_{n-1}, aPK_{n-1}, Mn) = 1}$.
    Vaults are updated such that ${Vault\{(E_{n-1}, aPK_{n-1})\} \coloneqq (e_n, aPK_n)}$.

    \item \textbf{Chain extension}.
    Every online consensus group member collects en existed chains and verifying that for every chain every block,\
    produced up to $Z$ blocks before contains correct data about slot $sl'$ leader $PK'$.
    To verify a valid slot leader, response from the ${\mathcal_{F}}_{VRF}$ to query\
    ${(\textbf{Verify}, sid, \eta' || sl' || \textbf{test}, y', \pi', v^{vrf'})}$ should be\
    ${(\textbf{Verified}, sid, \eta' || sl' || \textbf{test}, y', \pi', 1)}$ and $y'<T_j'$ as well.
    String \textbf{test} is an arbitrary and value $T_j'$ is a threshold for the stakeholder $PK'$ for the slot $sl'$.

    Then every lottery participant separately evaluates ${\mathcal_{F}}_{VRF}$ with his own inputs\
    ${(\textbf{EvalProve}, sid, \eta_j || sl || \textbf{nonce})}$ and\
    ${(\textbf{EvalProve}, sid, \eta_j || sl || \textbf{test})}$, where \textbf{nonce} is an arbitrary string.
    Received outputs ${(\textbf{Evaluated}, sid, y, \pi)}$ and ${(\textbf{Evaluated}, sid, \rho_y, \rho_\pi)}$\
    respectively includes generated random numbers ${y, \rho_y}$ and the associated proofs ${\pi, \rho_\pi}$.
    If ${y < T_i^j}$ then $PK_i$ is a slot leader and he can propose a batch which should be notarized by all\
    committee members.
    Random number $\rho_y$ will be used to sample a random seed for the next epoch.

    \textbf{Note:} The ${\mathcal_{F}}_{VRF}$ is designed in such a way that not every slot has a leader,\
    moreover, most of the slots remain empty to serve protocol synchronization.
    If there are $P$ several elected leaders for this slot, they all propose a new batches of updates\
    $\{B_p\}_{p=0}^P$ with included proofs of the leadership ${(v_i^{vrf}, y, \pi)}$ and ${(\pi, \rho_\pi)}$.

    After notarization batch is shared with over committees via syncing shards mechanism.
    After syncing any of the separate committee leaders can propose a block consisted of notarized batches and sign\
    it with ${\mathcal_{F}}_{KES}$ by sending the request ${(\textbf{sign\_req}, sid, PK_i, B_p, sl)}$.
    Received from ${\mathcal_{F}}_{KES}$ response signature $\sigma_p$ is included in the proposed block.

\end{enumerate}


        \section{Applications}\label{sec:applications}
        \subsection{Decentralized Cross-Chain Oracle}\label{subsec:cross-chain-oracle}

In oracle mode of operation the system is capable of providing a notarized set of events \
observed on supported blockchains.
Cross-Chain Oracle is a simple yet solving the cross-chain interoperability solution.

\subsection{Custodial Asset Management}\label{subsec:custodial-asset-management}

In custodial mode of operation the system is capable of managing user assets which are stored on corresponding \
blockchains in \emph{vaults}.
Each vault stores epoch number $n$, an aggregated public key $aPK_n$ of the current validator set $V_n$ and
is guarded with a script (smart-contract) capable of performing verification of
an aggregated signature ${verify: (\sigma_n, m_n, aPK_n) \rightarrow 0 | 1}$.

\paragraph{Natively Cross-Chain Applications}

Decentralized custodial management in conjunction with a computational layer can be highly beneficial for expanding \
the capabilities of the system.
Moving beyond simple bridges to what we call \emph{Natively Cross-Chain Applications} (NCCAs).
NCCAs are applications that are deployed in cross-chain network and are capable of interacting with other blockchains \
without the need of external oracles or bridges.


        \newpage
        \printbibliography

        \newpage
        \appendix
        \section*{Appendices}\label{sec:appendix}
        \addcontentsline{toc}{section}{Appendices}
        \renewcommand{\thesubsection}{\Alph{subsection}}
        \input{Appendix}

    \end{sloppypar}
\end{document}