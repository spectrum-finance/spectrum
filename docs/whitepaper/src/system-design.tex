This section presents Spectrum protocol design starting from a naive approach based on PBFT and gradually addressing the challenges.

\subsection{Strawman Design: PBFTNetwork}\label{subsec:strawman-design}

For simplicity we begin with a notarization protocol based on PBFT, then iteratively refine it into Spectrum.

PBFTNetwork assumes that a group of ${n = 3f + 1}$ trusted nodes has been pre-selected upfront and fixed and at most $f$ of these nodes are byzantine.
At any given time one of these nodes is the \emph{leader}, who observes events on connected blockchains,
batch them and initiate round of notarization within the consensus group.
Remaining members of the consensus group verify the proposed batches by checking the presence of updates on corresponding blockchains.
Upon successful verification each node signs the batch with its secret key and sends the signature to the leader.

Under simplifying assumptions that at most $f$ nodes are byzantine the PBFTNetwork guarantees livness and safety.
However, the assumption of a fixed trusted committee is not realistic for open decentralized systems.
Moreover, as PBFT consensus members authenticate each other via non-transferable symmetric-key MACs, each consensus
member has to communicate with others directly, what results in $O(n^2)$ communication complexity.
Quadratic communication complexity imposes a hard limit on scalability of the system.
Such a design also scales poorly in terms of adding support for more chains.
The workload of each validator grows lineary with each added chain.

In the subsequent sections we address these limitations in four steps:
\begin{enumerate}
    \item \textbf{Opening consensus group and leaders.} We introduce a lottery-based mechanism for selecting consensus group and leaders dynamically.
    \item \textbf{Replacing MACs by Digital Signatures.} We replace MACs by digital signatures to make authentication transferable
    and thus opening the door for sparser communication patterns that can help to reduce the communication complexity.
    \item \textbf{Scalable Collective Signature Aggregation.} We utilize Byzantine-tolerant aggregation protocol that allows for
    quick aggregation of cryptographic signatures to reduce communication complexity to $O(\log n)$.
    \item \textbf{Eliminating Validator Bottleneck.} We shard consensus groups into units by the type of chain each node is able to handle.
\end{enumerate}

\subsection{Opening Consensus Group}\label{subsec:opening-consensus-group-and-leaders}
Spectrum is an open-membership protocol, so PBFTNetwork's assumption on a closed consensus group is not valid.
Sybil attacks can break any protocol with security thresholds and an appropriate dynamic selection of\
the consensus group becomes crucial for preserving network's liveness and safety.
Election of consensus group members should be performed in a random and trustless way to ensure that\
a sufficient fraction (at most $f$ out of ${3 f + 1}$) of members are honest.

Similar selection mechanics is required in most blockchain protocols.
Bitcoin~\cite{nakamoto2009bitcoin} and many its successors are using Proof-of-Work (PoW) consensus,\
which, in essence, is a robust mechanism that facilitates randomized selection of a leader who is\
eligible to produce a new block.
Later, PoW approach was adapted into a Proof-of-Membership mechanism~\cite{kokoriskogias2016enhancing}.\
This mechanism allows once in a while to select a new consensus group\
which then executes the PBFT consensus protocol.

A primary consideration regarding PoW-based consensus mechanisms is\
the amount of energy required to operate such systems.
A natural alternative to PoW is a mechanism based on the concept of Proof-of-Stake (PoS)~\cite{King2012PPCoinPC}.
Rather than investing computational resources in order to participate in the leader selection process,\
participants of a PoS system instead run a process that randomly selects one of them proportionally to the stake.
Pure PoS mechanism to solve the PBFT problem was firstly used in~\cite{cryptoeprint:2017/454} to select both consensus\
group members and PBFT rounds leaders and to introduce randomness into this process,\
a verifiable Random Function (VRF) has been applied.

\subsubsection{Verifiable Random Function}

A Verifiable Random Function (VRF)~\cite{Micali1999} is a reliable way to introduce randomness into a protocol.
By definition, a function $\mathcal{F}$ can be attributed to the VRF family if the following methods are defined\
for the $\mathcal{F}$:
\begin{itemize}
    \item[--] Gen: ${Gen(1^l) \rightarrow (PK, SK)}$, where $PK$ is the public key and $SK$ is the secret key.
    \item[--] Prove: ${Eval(x, SK) \rightarrow \pi}$, where $x$ is an input and $\pi \vcentcolon= \Pi(x, SK)$ is\
    the proof, associated with $x$ and mixed with a random value, sampled from $\{0,1\}^{l_{\text{VRF}}}$.
    \item[--] Verify: ${Verify(x, \pi, PK) \rightarrow 0 | 1}$, where the output is $1$ if\
    and only if ${\pi \equiv \Pi(x, SK)}$.
\end{itemize}

The most secure implementations of VRF nowadays are Elliptic Curve Verifiable Random Functions (ECVRFs).
Basically, ECVRF is a cryptographic-based VRF that satisfies the uniqueness, collision resistance,\
and full pseudorandomness properties~\cite{cryptoeprint:2014/905}.
The security of ECVRF follows from the decisional Diffie-Hellman assumption in the random oracle model, thus\
ECVRF is a good source of randomness for a blockchain protocol.
Using ECVRF is also cheap and fast, since single ECVRF evaluation is approximately 100 microseconds on\
x86-64 for a specific curves used in hash functions.
Moreover, there is a great UC-extension for batch verification proposed by~\cite{cryptoeprint:2022/1045}\
which make it even faster by reducing the number of evaluations.

\subsubsection{Lottery}
Our lottery mechanism is based on ECVRF as a source of randomness and is generally inspired\
by Ouroboros Praos~\cite{cryptoeprint:2017/573} and Algorand~\cite{cryptoeprint:2017/454}.
The lottery mechanism in general allows the protocol assign a specific \emph{role} to a participant,\
while the validity of the participant's role can be verified using only publicly available data.

The main assigning logic is as follows:
\begin{legal}
    \item Participant calculates a certain threshold value $T$ according to predefined rules and\
    using only publicly available data for the calculation.
    \item Participant evaluates VRF function and calculates a random number $y$ using the VRF's proof $\pi$.
    \item If ${y < T}$ then the participant is considered valid for the respective role.
\end{legal}

To be more precise, let's clarify that in our setting a threshold value $T$ is calculated according\
to the formula ${T = 2^{l_{\text{VRF}}}\cdot \phi_{f}(\alpha, f)}$ where\
${\alpha=s/\\\sum_{i=1}^{M} s_i}$ is a relative stake.
Consequently, the probability of winning is calculated as ${p(\alpha, f) = 1-(1-f)^{\alpha}}$.
Thus, the winning probability depends on the participant's relative stake and is adjusted by the free parameter $f$.
This is where the PoS concept comes into play: the bigger the stake, the higher the chance of winning the lottery.

The lottery mechanism is fast, secure, and adaptive, since the involved pre-defined parameters\
can be changed via the voting process.
Moreover, the same primitives can be used to achieve different goals and we will utilize the lottery mechanism\
in several aspects of our protocol.

\textbf{Consensus Group Lottery}.
In the current section, we are considering a lottery mechanism application for \emph{dynamic consensus group selection}.
The Spectrum protocol initially is running by the manually selected opening consensus group $\{PK_i\}_{i=1}^M$\
of the predefined size $M$.
Stakeholders interact with each other and with locally installed ideal functionalities\
$\mathcal{F}_{\text{VRF}}$ and $\mathcal{F}_{\text{LB}}$ over a sequence of $L = E \cdot R$ slots\
${S=\{sl_1,\dots,sl_L\}}$ consisting of $E$ epochs with $R$ slots each.

Let's clarify what the mentioned above pre-defined primitives are needed for.
The ideal Verifiable Random Function functionality ${\mathcal{F}}_{\text{VRF}}$ we use here is similar to the extended VRF functionality\
introduced by Christian Badertscher et al.~\cite{cryptoeprint:2022/1045}:

Ideal Leaky Beacon functionality $\mathcal{F}_{\text{LB}}$ is used to sample an epoch random seed from the\
blockchain and is defined as follows:
\begin{functionality}
    \caption{${\mathcal{F}_{\text{LB}}(e_n, C_{\text{loc}})}$}
    \begin{algorithmic}
        \State \lstinline|// New epoch random seed is sampled once per epoch.|
        \State \lstinline|// C_loc is the local chain of the validator.|
        \If{${e_n < 2}$}
            \State \Return ${\textsf{false}}$
        \EndIf
        \For {each ${B_k \in C_{\text{loc}} \ |\ (B_k\textsf{.get(}e\textsf{)} \leq e_{n - 1})\ \wedge (\forall B_k\textsf{.get(}sl\textsf{)} \in R \cdot (n - 1) \cdot 2 /\\ 3)}$}
            \State \lstinline|// Every block B_k in the C_loc was produced by i'-th leader|
            \State \lstinline|// during j'-th slot, i.e. k = (i', j').|
            \State ${\pi^{\text{sl}} \leftarrow B\textsf{.get(}\pi^{\text{sl}}\textsf{)}}$
            \State ${r^{\text{sl}} \leftarrow \textsf{extract\_random(}\pi^{\text{sl}}\textsf{)}}$
            \State $y^{\text{rand}} \leftarrow \mathcal{H}(r^{\text{sl}} || \textsf{RAND})$
            \State ${\eta_n \leftarrow \mathcal{H}(\eta_{n - 1} || e_n || y^{\text{rand}})}$
        \EndFor
        \State \Return $\eta_n$
    \end{algorithmic}
\end{functionality}
An extended formal analysis of the security guaranties of the ${\mathcal{F}}_{\text{LB}}$ can be found in the original Ouroboros Praos\
paper~\cite{cryptoeprint:2017/573}.

Consensus group is constantly rotated each epoch ${e_n \gt 2}$.
Any verified protocol participant $PK_i$ can try to become a temporal member of the consensus group.
Participant is verified if his verification key tuple $v_i^{\text{ver}}$ is published in the blockchain during\
the epoch $e_{j-2}$ in the special $\textsf{VerificationRegTx}(v_i^{\text{ver}})$.
The consensus group lottery flow is as follows:
\begin{enumerate}
    \item At the end of the epoch ${e_n \gt 2}$ every verified participant $PK_i$ requests a\
    new epoch seed $\eta_n$ from the ${\mathcal{F}}_{\text{LB}}$.
    \item New consensus lottery threshold $T^{\text{cons}} = \phi_{f^{\text{cons}}}(\alpha^{n-2}_i)$\
    is calculated by every $PK_i$\ using stake distribution (to get the relative stake $\alpha^{n - 2}_{i}$)\
    from the blockchain state at the last block of the epoch $e_{n - 2}$.
    Free parameter $f^{\text{cons}}$ of the associated function $\phi$ is ${f^{\text{cons}} = M_n /\/ N_n}$,\
    where $M_n$ is a pre-defined number of new consensus group members to select at epoch $e_n$\
    and $N_n$ is the total number of verified stakeholders.
    \item When every $PK_i$ evaluates ${\mathcal{F}}_{\text{VRF}}$ with input\
    $x^{\text{cons}} = \eta_n || e_n $ and calculates the associated random number $y_{i, n}^{\text{cons}}$ from the received proof $\pi_{i, n}^{\text{e}}$, i.e.\
    ${y_{i, n}^{\text{cons}} = \mathcal{H}(r_{i, n}^{\text{e}}||\textsf{CONS})}$, where $r_{i, n}^{\text{e}}$\
    is a random number extracted from the proof and $\textsf{CONS}$ is an arbitrary pre-defined constant.
    \item To reveal the result of the consensus group lottery $PK_i$ compares value $y_{i, n}^{\text{cons}}$\
    with the threshold $T_{i, n}^{\text{cons}}$.
    If ${y_{i, n}^{\text{cons}} < T_{i, n}^{\text{cons}}}$\
    then the participant is a legal member of new consensus group which will be active in the epoch $e_{n+2}$.
    \item Finally, to declare his right to participate in the new consensus group, participant $PK_i$\
    includes an associated proof $\pi_{i, n}^{\text{e}}$ into the\
    $\textsf{ConsLotteryResTx}(e_n, v_i^{\text{vrf}}, \pi_{i, n}^{\text{e}})$ and adds it into the main chain.
\end{enumerate}
Note, that the members of the consensus group should be known ahead of time for the synchronization.
Therefore, in order to participate in the $e_n$ consensus lottery already verified participant must\
publish $\textsf{VerificationUpdTx}$ message at the epoch $e_{n-2}$.
Public disclosure of the future consensus group doesn't give much advantage to an adversary\
since there are hundreds of consensus members in every epoch and denial of service attacks are difficult to succeed.
At the same time any grinding attacks are limited because an adversary can't arbitrarily control $\eta_n$ values.

The main task of the validators set elected via the consensus group lottery is to observe and notarize\
events using a digital signature aggregation mechanism which we will introduce in the next sections.


\subsection{Replacing MACs by Digital Signatures}\label{subsec:replacing-macs-by-digital-signatures}

todo

\subsection{Scalable Collective Signature Aggregation}\label{subsec:scalable-collective-signature-aggregation}

todo

\subsection{Eliminating Validator Bottleneck}\label{subsec:eliminating-validator-bottleneck}

So far each member of consensus group had to track changes on all connected chains in order to participate in consensus properly.

\textbf{Observation 1:} Events coming from independent systems $S_k$ are not serialized.
Thus, the process of events notarisation can be parallelized.

\textbf{Observation 2:} Outbound transactions on independent systems $S_k$ can be independently signed.

Utilizing those properties we now introduce committee sharding.
We modify protocol in a way such that at each epoch $e$ $M$ distinct committees consisting of nodes equipped with functionality unit $F_{S_k}$ relevant to a specific connected chain $S_k$ are selected in a way described in (5.2.2).
All primitives and source of randomness are equal to different committees, the only difference is in the $f$ parameter of $\phi(\alpha_i, f)$ function, which is unique for every connected blockchain in order to guaranty expected number of members in every committee.
We denote one such committee shard as $V^{e}_{S_k}$, which uniquely maps to $S_k$.
Then, complete mapping of committees to chains at epoch $e$ can be represented as a set of tuples commettee-chain $\{(V^{e}_{S_k}, S_k)\}$.
Throughout epoch $e$ all events and on-chain transactions on $S_k$ are handled exclusively by $V^{e}_{S_k}$.

Nodes in $V^{e}_{S_k}$ maintain a robust local ledger $L^{local}_k$ of notarized batches of events observed in $S_k$.

\subsubsection{Syncing Shards}

Notarized batches of events from local ledgers $\{L_i\}_{i=1}^{i=N}$ then should be synced in a super ledger $L^+$ in order for the system to be able to compute cross-chain state transition.
To facilitate this process batches of notarized events are broadcast to other committees.
The main actors at this stage are:
\begin{itemize}
    \item \textbf{Local leaders}: committees leaders, holding local notarized batches.
    \item \textbf{Relayers}: any protocol participant, who broadcasts notarized batches from \emph{Local leaders} to other committees' members.
    Every \emph{Local leader} can be a \emph{Relayer} at the same time.
    \item \textbf{General leader}: one of the \emph{Local leaders} who added a block consisted of all collected notarized batches to the $L^+$.
\end{itemize}

Since any \emph{Local leader} is able to publish his block to $L^+$ he can choose from two main strategies:
\begin{itemize}
    \item \textbf{Wait}: malicious strategy where \emph{Local leader} waits for broadcasts from other committees members and don't broadcast his own batch to eliminate competitors for adding a block.
    \item \textbf{Broadcast and wait}: fair strategy where \emph{Local leader} immediately broadcasts his batch, waits for broadcasts from committees members and honestly competes for adding a block.
\end{itemize}
Thus, there should be a motivation for individual \emph{Local leader} to choose the fair strategy instead of keeping his batch for too long.
This is achieved through the design of the incentive system.

There are three types of incentive: ${\{R_b, R_d, R_m\}}$, where $R_b$ is a guaranteed reward for adding a notarized batch to the block, $R_d$ is given for a broadcasting batch to the general leader and $R_m$ is given personally to the \emph{General leader} who mined the block.
Delivery reward $R_d$ is given to the \emph{Relayer} if and only if a delivery was made within a predetermined period of time $\Delta t_d$.
From the game-theoretic analysis, the following relationships between rewards were derived: ${R_b = 2 \cdot R_d, R_m = 3 \cdot R_d}$.
Thus, if ${R_d=0}$ there is no prior strategy for the \emph{Local leaders}, they will or wait for other batches either broadcast their batches with equal probability.
In case when ${R_d>0}$ it is distributed between the \emph{Local leader} and the \emph{Relayer}, i.e. ${R^l_d = \xi \cdot R_d}$ and  ${R^r_d = (1 - \xi) \cdot R_d}$, where ${\xi \in (0, 1)}$.
While ${\xi \rightarrow 1}$ probability that all \emph{Local leaders} will choose the \emph{Broadcast and wait} strategy approaches $1$.

As a result, the syncing Shards flow looks as follows:
\begin{itemize}
    \item Every \emph{Local leader} broadcasts (himself or through an intermediary as a \emph{Relayer}) his batch $b_i$, which contains the local notarization time $t^N_i$ and waits for batches from other \emph{Local leaders}.
    \item When waiting time approaches $\Delta t_d$, \emph{Local leader} forms a block from all collected batches ${\{b_i^j\}_{j=1}^{j=K}, K \le N}$ and add it to $L^+$.
    Block contains the set of the notarization times $\{t^{N^j}_i\}_{j=1}^{j=K}$ and block creation time $t^B_s$.
    \item After block is settled, all associated actors receive their rewards according to their roles: \emph{General leader} receives $R_m$, \emph{Local leaders}, whose batches are in the block receives $R_b$.
    In addition, if ${t^B_s - t^N_i^* < \Delta t_d}$, where $t^N_i^*$ is $t^N_i$ time, normalized to $L^+$ time, $i$-th committee \emph{Local leader} receives $R_d$ reward shared with the \emph{Relayer}.
\end{itemize}

\subsection{Forks and integrity}\label{subsec:resolving-forks}
Protocol flow implies that any of the local leaders can append their blocks to $L^+$, which leads to forks.

This type of fork is a normal part of the protocol lifecycle, however, total possible number of the normal forks in our protocol is much larger than in other blockchains, since there can also be a several local leaders in every connected $L_i$ committee.
The chance of occurring a malicious forks produced by adversary is minimized by lottery design.
In addition, the task for an adversary becomes more difficult by virtue of the interaction between the protocol participants during the Syncing Shards process.

The main rules for resolving forks are simple and are performed by the members of all committees when validating a proposed blocks:
\begin{itemize}
    \item \textbf{Max valid}: choose the longest appropriate chain given a set of valid chains that are available in the network.
    The depth of any block broadcast by a protocol member during the protocol must exceed the depths of any honestly-generated blocks from slots at least $K$ in the past.
    \item \textbf{Max stake}: if the \emph{Max valid} rule doesn't resolve a slot battle, then the valid chain chooses according to the stake size of the battled leaders, the maximum stake is the winner.
\end{itemize}

A large number of the normal forks, however, still significantly affect properties, that maintain the integrity of the $L^+$:
\begin{itemize}
    \item \textbf{Latency}: the number of elapsed slots required for a transaction to appear in a block on the $L^+$.
    \item \textbf{Finality}: the number of elapsed slots required for a transaction to become settled and immutable.
\end{itemize}
The Latency of the protocol is good enough due to the short duration of the slots.
Finality is guaranteed after $K_F$ slots, where $K_F$ is a pre-defined protocol parameter.
As a result of the functional features of our protocol, $F_F$ depends on the connected $L_i$ integrity properties.

Most ledgers do not guarantee instant finality of transaction, that means that any (or all) transactions may not be applied to corresponding $L_i$ ledgers in the end.
Different blockchains however has different Finality parameters, and time of elapsing $K_F$ should be longer than all of them.
Thus, the $K_F$ should be set with a margin and therefore using the number of slots $\Delta Sl$ that have passed in the Spectrum network, developers should be able to receive information about the number of blocks that have passed in all connected $L_i$ blockchains during this period of time.
The duration of the block in each $L_i$ is different, but the average values are preserved for a certain period of time ${\Delta T >> d_s}$, where $d_s$ is the duration of Spectrum's slot.
Thus, after each $\Delta T$ time interval, Spectrum network will update the set of constants: ${(\{d_{i}\}_{i=1}^{M},\{c_{i}\}_{i=1}^{M})}$, where $d_i$ is a block duration in the $L_i$, $c_i$ is the default reliable number of confirmations in the $L_i$, $M$ is the total number of the connected $L_i$.

Using the data above, each Spectrum's $\Delta Sl$ can be associated with the delta of blocks that have passed in any connected blockchain: ${\{\lfloor \Delta Sl \cdot d_s \mathbin{/} d_i)\rfloor\}_{i=1}^{N}}$.
When forming transaction, developers can specify a reliability factor $C$.
This factor will be compared with the ratio of the number of blocks passed on the associated $L_i$ blockchain to the default reliable number of confirmations $c_i$ of this network:
\begin{equation}
    \theta(i-L_i^{id})\cdot \left\{\frac{1}{c_i} \cdot \left\lfloor \Delta Sl \cdot \frac{d_s}{d_i}\right\rfloor\right\}_{i=1}^{M} >= C,\label{eq:equation2}
\end{equation}
where $\theta(x)$ is an indicator function which is 1 at $x = 0$, otherwise 0.



\subsection{Protocol Flow}\label{subsec:protocol-flow}

\subsubsection{Bootstrapping}\label{subsubsec:bootstrapping}

The system is bootstrapped in a trusted way.
A manually picked set of validators $V_0$ is assigned to the first epoch $e_0$.
On-chain vaults are initialized with an aggregated public key $aPK_0$ of the initial committee.
All initial committee members generate verification tuples ${(v_i^{vrf}, v_i^{kes}, v_i^{dsig})}$
and agree on the genesis block.

\subsubsection{Normal Flow}\label{subsubsec:normal-flow}

\begin{enumerate}
    \item Registration.
    All Spectrum stakeholders can register for becoming a committee member.
    To get a chance of becoming a member of $V_n$ in the epoch $e_n$ they register in a lottery during the $e_{n-2}$
    epoch by publishing their verification tuples ${(v_i^{vrf}, v_i^{kes}, v_i^{dsig})}$.
    \item Lottery.
    Once registration is done and epoch $e_{n-1}$ comes to the end, all registered participants evaluates
    ${\mathcal_{F}}_{VRF}$ locally and compare the generated random $y$ with their corresponding consensus threshold
    ${T_i^j}^*$ for this epoch.
    If successful, then publish $y$ and the associated proofs to form an approved consensus members table.
    \item Committee key aggregation.
    Once new committee is selected, nodes in $V_n$ aggregate their individual public keys $\{PK_i\}$ into
    a joint one $aPK_n$.
    \item Committee transition.
    Nodes in $V_{n-1}$ publish cross-chain message ${M_n : (aPK_n, \sigma_{n-1})}$ , where $aPK_n$ is
    an aggregated public key of the new committee $V_n$ , $\sigma_{n-1}$ is an aggregated signature of
    $M_n$ such that ${Verify(\sigma_{n-1}, aPK_{n-1}, Mn) = 1}$.
    Vaults are updated such that ${Vault\{(E_{n-1}, aPK_{n-1})\} \coloneqq (e_n, aPK_n)}$.
    \item Decentralized Asset Management (Custodial).
    Nodes in $V_n$ observe events on supported L1 chains, agree on the set of updates
    and compute state outbound state transitions accordingly.
    \item Notarisation (Non-custodial).
    Nodes in $V_n$ observe events on supported L1 chains, batch updates, collectively sign them and
    publish on-chain.
\end{enumerate}