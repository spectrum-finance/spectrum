Following the success of Bitcoin, many blockchain-based cryptocurrencies have been developed and deployed.
To meet different requirements in various scenarios, a great number of heterogeneous blockchains have emerged.
However, most of the presented blockchain platforms are developed independently, therefore, they are\
designed for their own use cases and incompatible with each other.
Hence, interoperability between blockchains has become one of the key issues\
which prevents blockchain technology from wide adoption.

With fair blockchain interoperability users can potentially conduct transactions across different blockchain networks\
smoothly and without any intermediaries.
This guarantees a reduction in the fragmentation of the crypto ecosystem and opens up new horizons and business models.
Implementation of the blockchain interoperability protocol is challenging since different blockchains have\
different security solutions, consensus algorithms and programming languages.
An inaccurate solution can potentially increase the possibility of attacks and create management challenges\
across different connected networks.

The classic cross-chain interoperability solution is a trusted oracle that registers some event on one blockchain\
and performs the required action on the other.
Centralized oracles provide fast and cheap transactions but lack a key feature -- decentralization.
The liquidity of the protocol built on this approach is custodial which is a centralized approach similar to CeFi when\
users deposit their funds to an exchange's wallet.

Another common approach involves intermediate network consisting of a fixed number of hand-picked oracles to facilitate\
data transfer among multiple blockchains.
The consensus mechanism in such protocols is usually proof-of-authority or proof-of-stake, hence, the wide range\
of potential validators are eliminated due to verification procedures or high collateral and network moderation\
typically carried out by several dozen of rarely alternating nodes.
Moreover, a common practice is to store funds transferred between blockchains on some kind of threshold wallets,\
which are generated by the participants of the intermediate network.
This results in all funds being controlled by a fixed group of oracle operators.
Such a system is also not truly decentralized.

Regarding the application scenarios, one of the most popular in the existing\
blockchain interoperability proposals is an atomic token swap.
However, atomic token swapping protocols~\cite{Miraz2019} are not self-inclusive enough to complete the tasks of\
cross-chain decentralized applications with complex activities than just a token exchanges.
The reason is that the atomic swapping process does not have the ability to destroy a certain amount\
of assets in the source blockchain and re-create the same amount on the target blockchain.
Moreover, this process always requires a counterparty who is willing to exchange tokens~\cite{Schulte2019TowardsBI}.

True blockchain interoperability requires the users and developers have the ability to access information\
from one blockchain inside another without any additional efforts from a third party.
This is a great-efforts task, thus, before achieving a successfully interoperable multi-blockchain system,\
many challenges must be overcome, such as scalability when applying to a large-scale scenario~\cite{Kim2018} and etc.

The motivation of this paper is to describe the Spectrum protocol, which provides an open, truly decentralized,\
secure and scalable cross-chain interoperability solution.
The Spectrum protocol is intended for both end-users and developers, who will be able\
to implement their applications on top of the protocol to widespread\
the use of blockchain technology in various business areas.
