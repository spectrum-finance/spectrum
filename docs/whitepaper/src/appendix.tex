Validators in the Spectrum network consist of several committees, individually selected via the lottery mechanism.
Number of committees is equal to the number of the currently connected $L_1$ blockchains.

The exact number of members in each committee may vary, but there are always hundreds of active validators for every\
local chain.
We will denote a committee with $\mathbf{C}_i$, where $i$ is the index of the certain $c_i$ connected chain.
At any given moment of time each committee $\mathbf{C}_i$ observes events on the local $c_i$ and\
on the Spectrum ledger $L^+$.
Assets are managed by the Network are stored on-chain in vaults.
Each asset type is controlled by local committees using via the aggregated signature mechanism.

\subsubsection{Processing events and transactions}

The main feature of the Spectrum is a native cross-chain interoperability, thus, more complex transactions take\
place here than on other networks.
Every committee of validators observes events in the local chains and in the Spectrum network.
There are 3 types of possible events coming from the external systems:
\begin{enumerate}
    \item \textbf{Inbound Value Transfer}: a box appeared in the local chain $c_i$, the equivalent value of which\
    should be transferred to the Spectrum network;
    \item \textbf{Outbound Value Transfer}: transfer of value from the Spectrum network into the external\
    system $c_i$ should be certified by the relevant local committee $\mathbf{C}_i$;
    \item \textbf{Box Elimination}: outbound value is reliably settled on the external chain $c_i$ and the\
    corresponding box in the Spectrum network must be eliminated.
\end{enumerate}

Incoming events from every $c_i$ are grouped into batches and executed independently\
by each local committee $\mathbf{C}_i$.
Therefore, the design of the Spectrum network provides 2 global types of transactions:
\begin{enumerate}
    \item \textbf{Batch Applying Transaction}: a transaction consisting of operations corresponding to the\
    execution of the incoming events: \textit{Inbound Value Transfers}, \textit{Outbound Value Transfers}\
    and \textit{Boxes Eliminations};
    \item \textbf{Programmable Transaction}: any programmable transaction within the Spectrum network.
\end{enumerate}

Spectrum ledger $L_s$ is based on the eUTXO model, we will denote transactions in the usual manner:
\begin{equation}
{Tx_{(in | out)}[Spent\ Boxes \rightarrow Unspent \ Boxes]}
    ,
\end{equation}
where ${in, out}$ stands for \textit{inbound} or \textit{outbound} transaction type respectively.
Box designations can be as follows: ${Box_{owner}^{(s | l | e)}(value)}$, where ${s, l, e}$ stands\
for the Spectrum box status.
Designation $s$ is \textit{available} Spectrum box, $l$ is \textit{locked} Spectrum box and $e$ is\
\textit{eliminating} Spectrum box.
Moreover, due to the design of the system, boxes can have different status that determine\
the range of actions that are valid for them.

Let's take an example of the cross-chain swap, which will allow us to distinctly follow\
the statuses and limitations follows from the above.

Alice wants to swap her assets from the blockchain $c_1$ to the blockchain $c_2$ using the Spectrum network\
and she has valid addresses on all networks.
Let's also assume that there is a box ${Box_{S}^s(X_{c_2})}$ available for spending in the liquidity of the\
Spectrum network ($S$ is the owner, i.e. Spectrum).
The swap flow will be as follows:
\begin{enumerate}

    \item Alice transfer her assets $X_{c_1}$ into the Spectrum vault an the inbound box ${Box_{A}(X_s)}$ with\
    the corresponding value $X_s$ is formed.
    Committee $\mathbf{C}_1$ observes this \textit{Inbound Value Transfer} event and\
    includes it into the \textit{Batch Applying Transaction}.

    \item Committee $\mathbf{C}_1$ processes the \textit{Batch Applying Transaction} and after adding\
    the block into the $L^+$ the box $Box_{A}^l(X_s)$ appears in the Spectrum network in the \textit{locked status}.
    The locked status remains until the time corresponding to the reliability confirmation threshold $R_1$ on\
    the network $c_1$ has passed.

    \item Once $R_1$ time is passed and status of the box is changed to $Box_{A}^s(X_s)$,\
    the \textit{Outbound Value Transfer} event is formed to transfer the $X_{c_2}$ value to Alice on the chain $c_2$.
    This event is observed by the committee $\mathbf{C}_2$ and is included in their \textit{Batch Applying Transaction}.
    After processing the $X_{c_2}$ value is transferred from the ${Box_{S}^s(X_{c_2})}$ according to the generated\
    certificate and the $Box_{A}^e(X_s)$ goes to the \textit{eliminating status}.
    This status will be preserved until enough time corresponding to the reliability confirmation\
    threshold $R_2$ on the network $c_2$ has passed.

    \item Once $R_2$ time is passed, the \textit{Box Elimination} event is formed,\
    observed by the $\mathbf{C}_2$, included into the \textit{Batch Applying Transaction} and processed.
    The box $Box_{A}^e(X_s)$ is eliminated from the next block and cross-chain swap is considered successfully completed.

\end{enumerate}

\subsubsection{Consensus assumptions}
It follows from the above description that in order to achieve consensus in cross-chain translation,\
synchronization between all committees is necessary only for certain actions, thus:
\begin{itemize}
    \item \textbf{Global Synchronization}: is needed to process \textit{Programmable Transactions} and to update\
    the status of boxes in the Spectrum network, i.e. \textit{locked} to \textit{available}, \textit{available} to\
    \textit{eliminating} or \textit{eliminating} to \textit{available};
    \item \textbf{Local Synchronization}: is needed between local committee to process the\
    \textit{Batch Applying Transaction}, this type of transaction can be processed separately by all committees.
\end{itemize}